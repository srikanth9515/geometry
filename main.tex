%% Run LaTeX on this file several times to get Table of Contents,
%% cross-references, and citations.

\documentclass[11pt]{book}
\usepackage{gvv-book}
\usepackage{gvv}
%\usepackage{Wiley-AuthoringTemplate}
\usepackage[sectionbib,authoryear]{natbib}% for name-date citation comment the below line
%\usepackage[sectionbib,numbers]{natbib}% for numbered citation comment the above line

%%********************************************************************%%
%%       How many levels of section head would you like numbered?     %%
%% 0= no section numbers, 1= section, 2= section, 3= subsection %%
\setcounter{secnumdepth}{3}
%%********************************************************************%%
%%**********************************************************************%%
%%     How many levels of section head would you like to appear in the  %%
%%				Table of Contents?			%%
%% 0= chapter, 1= section, 2= section, 3= subsection titles.	%%
\setcounter{tocdepth}{2}
%%**********************************************************************%%

%\includeonly{ch01}
\makeindex

\begin{document}

\frontmatter
%%%%%%%%%%%%%%%%%%%%%%%%%%%%%%%%%%%%%%%%%%%%%%%%%%%%%%%%%%%%%%%%
%% Title Pages
%% Wiley will provide title and copyright page, but you can make
%% your own titlepages if you'd like anyway
%% Setting up title pages, type in the appropriate names here:

\booktitle{Geometry}

\subtitle{Through Trigonometry}

\AuAff{G. V. V. Sharma}


%% \\ will start a new line.
%% You may add \affil{} for affiliation, ie,
%\authors{Robert M. Groves\\
%\affil{Universitat de les Illes Balears}
%Floyd J. Fowler, Jr.\\
%\affil{University of New Mexico}
%}

%% Print Half Title and Title Page:
%\halftitlepage
\titlepage

%%%%%%%%%%%%%%%%%%%%%%%%%%%%%%%%%%%%%%%%%%%%%%%%%%%%%%%%%%%%%%%%
%% Copyright Page

\begin{copyrightpage}{2022}
%Title, etc
\end{copyrightpage}

% Note, you must use \ to start indented lines, ie,
% 
% \begin{copyrightpage}{2004}
% Survey Methodology / Robert M. Groves . . . [et al.].
% \       p. cm.---(Wiley series in survey methodology)
% \    ``Wiley-Interscience."
% \    Includes bibliographical references and index.
% \    ISBN 0-471-48348-6 (pbk.)
% \    1. Surveys---Methodology.  2. Social 
% \  sciences---Research---Statistical methods.  I. Groves, Robert M.  II. %
% Series.\\

% HA31.2.S873 2004
% 001.4'33---dc22                                             2004044064
% \end{copyrightpage}

%%%%%%%%%%%%%%%%%%%%%%%%%%%%%%%%%%%%%%%%%%%%%%%%%%%%%%%%%%%%%%%%
%% Only Dedication (optional) 

%\dedication{To my parents}

\tableofcontents

%\listoffigures %optional
%\listoftables  %optional

%% or Contributor Page for edited books
%% before \tableofcontents

%%%%%%%%%%%%%%%%%%%%%%%%%%%%%%%%%%%%%%%%%%%%%%%%%%%%%%%%%%%%%%%%
%  Contributors Page for Edited Book
%%%%%%%%%%%%%%%%%%%%%%%%%%%%%%%%%%%%%%%%%%%%%%%%%%%%%%%%%%%%%%%%

% If your book has chapters written by different authors,
% you'll need a Contributors page.

% Use \begin{contributors}...\end{contributors} and
% then enter each author with the \name{} command, followed
% by the affiliation information.

% \begin{contributors}
% \name{Masayki Abe,} Fujitsu Laboratories Ltd., Fujitsu Limited, Atsugi, Japan
%
% \name{L. A. Akers,} Center for Solid State Electronics Research, Arizona State University, Tempe, Arizona
%
% \name{G. H. Bernstein,} Department of Electrical and Computer Engineering, University of Notre Dame, Notre Dame, South Bend, Indiana; formerly of
% Center for Solid State Electronics Research, Arizona
% State University, Tempe, Arizona 
% \end{contributors}

%%%%%%%%%%%%%%%%%%%%%%%%%%%%%%%%%%%%%%%%%%%%%%%%%%%%%%%%%%%%%%%%
% Optional Foreword:

%\begin{foreword}
%\lipsum[1-2]
%\end{foreword}

%%%%%%%%%%%%%%%%%%%%%%%%%%%%%%%%%%%%%%%%%%%%%%%%%%%%%%%%%%%%%%%%
% Optional Preface:

%\begin{preface}
%\lipsum[1-1]
%\prefaceauthor{}
%\where{place\\
% date}
%\end{preface}

% ie,
% \begin{preface}
% This is an example preface.
% \prefaceauthor{R. K. Watts}
% \where{Durham, North Carolina\\
% September, 2004}

%%%%%%%%%%%%%%%%%%%%%%%%%%%%%%%%%%%%%%%%%%%%%%%%%%%%%%%%%%%%%%%%
% Optional Acknowledgments:

%\acknowledgments
%\lipsum[1-2]
%\authorinitials{I. R. S.}  

%%%%%%%%%%%%%%%%%%%%%%%%%%%%%%%%
%% Glossary Type of Environment:

% \begin{glossary}
% \term{<term>}{<description>}
% \end{glossary}

%%%%%%%%%%%%%%%%%%%%%%%%%%%%%%%%
%\begin{acronyms}
%\acro{ASTA}{Arrivals See Time Averages}
%\acro{BHCA}{Busy Hour Call Attempts}
%\acro{BR}{Bandwidth Reservation}
%\acro{b.u.}{bandwidth unit(s)}
%\acro{CAC}{Call / Connection Admission Control}
%\acro{CBP}{Call Blocking Probability(-ies)}
%\acro{CCS}{Centum Call Seconds}
%\acro{CDTM}{Connection Dependent Threshold Model}
%\acro{CS}{Complete Sharing}
%\acro{DiffServ}{Differentiated Services}
%\acro{EMLM}{Erlang Multirate Loss Model}
%\acro{erl}{The Erlang unit of traffic-load}
%\acro{FIFO}{First in - First out}
%\acro{GB}{Global balance}
%\acro{GoS}{Grade of Service}
%\acro{ICT}{Information and Communication Technology}
%\acro{IntServ}{Integrated Services}
%\acro{IP}{Internet Protocol}
%\acro{ITU-T}{International Telecommunication Unit -- Standardization sector}
%\acro{LB}{Local balance}
%\acro{LHS}{Left hand side}
%\acro{LIFO}{Last in - First out}
%\acro{MMPP}{Markov Modulated Poisson Process}
%\acro{MPLS}{Multiple Protocol Labeling Switching}
%\acro{MRM}{Multi-Retry Model}
%\acro{MTM}{Multi-Threshold Model}
%\acro{PASTA}{Poisson Arrivals See Time Averages}
%\acro{PDF}{Probability Distribution Function}
%\acro{pdf}{probability density function}
%\acro{PFS}{Product Form Solution}
%\acro{QoS}{Quality of Service}
%\acro{r.v.}{random variable(s)}
%\acro{RED}{random early detection}
%\acro{RHS}{Right hand side}
%\acro{RLA}{Reduced Load Approximation}
%\acro{SIRO}{service in random order}
%\acro{SRM}{Single-Retry Model}
%\acro{STM}{Single-Threshold Model}
%\acro{TCP}{Transport Control Protocol}
%\acro{TH}{Threshold(s)}
%\acro{UDP}{User Datagram Protocol}
%\end{acronyms}

\setcounter{page}{1}

\begin{introduction}
This book shows how to solve problems in geometry using trigonometry. 

\end{introduction}

\mainmatter

%

\chapter{Triangle}
\input{./chapters/exercises/tri_geo_exer}
%
\chapter{Quadrilateral}
\input{./chapters/exercises/quad_geo_exer}
%
\chapter{Circle}
\input{./chapters/exercises/circ_geo_exer}
\chapter{Miscellaneous }
\input{./chapters/exercises/geo_misc}

\iffalse
\section{Quadrilateral Constructions}
\input{./chapters/exercises/quad_geo_const}
\section{Circle Constructions}
\input{./chapters/exercises/circ_const_exer}
\section{Triangle Constructions}
\input{./chapters/exercises/tri_geo_const}
\chapter{Two Dice}
\\ \solution
%
The intersection of 
		\eqref{eq:geo-alt-be}
		and
		\eqref{eq:geo-alt-cf},
		is obtained from 
		the matrix equation
		%
\begin{align}
        \myvec{1&1\\5&-7} \vec{x} &= \myvec{2\\20}
\end{align}
%
which can be solved as 
%
\begin{align}
        \myvec{1&1&2\\5&-7&20}
	 \xleftrightarrow[]{R_2 \leftarrow R_2 - 5R_1}
        \myvec{1&1&2\\0&-12&10}\\
	 \xleftrightarrow[]{R_2 \leftarrow \frac{R_2}{-12}}
        \myvec{1&1&2\\0&1&\frac{-5}{6}}
	 \xleftrightarrow[]{R_1 \leftarrow R_1 - R_2}
        \myvec{1&0&\frac{17}{6}\\0&1&\frac{-5}{6}}
\end{align}
%
yielding
%
\begin{align}
        \vec{H}&=\frac{1}{6}\myvec{{17}\\{5}}
		\label{eq:geo-alt-H},
\end{align}
%
See 
\figref{fig:m_tri_py}
\begin{figure}[!ht]
\centering
\includegraphics[width=\columnwidth]{solutions/1/3/4/figs/Figure_1.png}
\caption{Intersection point $\vec{H}$ of altitudes B$E_{1}$ and C$F_{1}$ plotted using python}
\label{fig:m_tri_py}
\end{figure}


\fi

%\include{ch02} 
\backmatter
\appendix
\chapter{Baudhayana Theorem}
\section{The Right Angled Triangle}
A right angled triangle looks like Fig. \ref{fig:tri_right_angle}.
\begin{figure}[!ht]
\centering
\resizebox{\columnwidth}{!}{\input{./figs/triangle/tri_right_angle.tex}}
\caption{Right Angled Triangle}
\label{fig:tri_right_angle}	
\end{figure}
with angles $\angle A,\angle B$ and $\angle C$ and sides $a, b$ and $c$.  The unique feature of this triangle is $\angle B$ which is defined to be $90\degree$.
%\renewcommand{\theequation}{\theenumi}
\begin{enumerate}[label=\thesection.\arabic*.,ref=\thesection.\theenumi]
\numberwithin{equation}{enumi}
\item
	For simplicity, let the greek letter $\theta = \angle C$.  We have the following definitions.
\begin{equation}
\label{eq:tri_trig_defs}
\begin{matrix}
	\sin \theta = \frac{c}{b} & 	\cos \theta = \frac{a}{b} \\
	\tan \theta = \frac{c}{a} & \cot \theta = \frac{1}{\tan \theta} \\
	\csc \theta = \frac{1}{\sin \theta} & \sec \theta = \frac{1}{\cos \theta}
	\end{matrix}
\end{equation}
%
\item  Show that
	\begin{equation}
	\cos \theta = \sin \brak{90\degree - \theta}
	\label{eq:tri_baudh_comp}	
	\end{equation}
\solution From \eqref{eq:tri_trig_defs},
%
\begin{align}
\label{eq:tri_90-ang}
\cos \angle BAC = \cos \alpha =	\cos \brak{90\degree-\theta} = \frac{c}{b} 
%\\
= \sin \angle ABC = \sin \theta
\end{align}
\item
In Fig. \ref{fig:tri_cosine_formula}, show that
%
\begin{equation}
\label{eq:tri_cos_form}
\cos A = \frac{b^2+c^2-a^2}{2bc}
\end{equation}
%
\begin{figure}[!ht]
	\begin{center}
		
		%\includegraphics[width=\columnwidth]{./figs/ch2_triang_ar}
		%\vspace*{-10cm}
		\resizebox{\columnwidth}{!}{\input{./figs/triangle/tri_cosine_formula.tex}}
	\end{center}
	\caption{The cosine formula}
	\label{fig:tri_cosine_formula}	
\end{figure}
\solution From Fig. \ref{fig:tri_cosine_formula}, 
%
\begin{align}
a &= x + y = b \cos C + c \cos B.
\end{align}
%
Similarly,
%
\begin{align}
b &= c \cos A + a \cos C \\
c &= b \cos A + a \cos B
\end{align}
%
The above equations can be expressed in matrix form as
%
\begin{equation}
\begin{pmatrix}
0 & c & b \\
c & 0 & a \\
b & a & 0
\end{pmatrix}
\begin{pmatrix}
\cos A \\
\cos B \\
\cos C
\end{pmatrix}
= 
\begin{pmatrix}
a\\
b\\
c
\end{pmatrix}
\end{equation}
%
Using the properties of determinants,
%
\begin{align}
\cos A = \frac{
\begin{vmatrix}
a & c & b \\
b & 0 & a \\
c & a & 0
\end{vmatrix}
	}
	{
\begin{vmatrix}
0 & c & b \\
c & 0 & a \\
b & a & 0
\end{vmatrix}
	}
	=\frac{ab^2 + ac^2 - a^3}{abc + abc} 
= \frac{b^2 + c^2 - a^2}{2abc}
\end{align}
\iffalse
\item Draw Fig. \ref{fig:tri_right_angle} for $a = 4, c =3$.
\label{const:tri_right_angle}
%
\\
\solution The vertices of $\triangle ABC$ are 
\begin{align}
\vec{A} = \myvec{0\\c} = \myvec{0\\3}, \vec{B} = \myvec{0\\0}, \vec{C} = \myvec{a\\0}=\myvec{4\\0}
\end{align}
%
The python code for  Fig. \ref{fig:tri_right_angle} is
\begin{lstlisting}
codes/triangle/tri_right_angle.py
\end{lstlisting}
%
and the equivalent latex-tikz code is
%
\begin{lstlisting}
figs/triangle/tri_right_angle.tex
\end{lstlisting}
%
The above latex code can be compiled as a standalone document as
%
\begin{lstlisting}
figs/triangle/tri_right_angle_alone.tex
\end{lstlisting}
%
\item Draw Fig. \ref{fig:tri_polar} for $a = 4, c =3$.
\label{const:tri_polar}
%
\\
\solution The vertices of $\triangle ABC$ are 
\begin{align}
\vec{A} = \myvec{a\\c} = \myvec{4\\3}, \vec{B} = \myvec{a\\0}  = \myvec{4\\0}, \vec{C} = \myvec{0\\0}.
\end{align}
%
The python code for  Fig. \ref{fig:tri_polar} is
\begin{lstlisting}
codes/triangle/tri_polar.py
\end{lstlisting}
%
and the equivalent latex-tikz code is
%
\begin{lstlisting}
figs/triangle/tri_polar.tex
\end{lstlisting}
\begin{figure}[!ht]
\centering
\resizebox{\columnwidth}{!}{\input{./figs/triangle/tri_polar.tex}}
\caption{Right Angled Triangle}
\label{fig:tri_polar}	
\end{figure}
%
\item The vertex  $\vec{A}$ can also be expressed  in {\em polar coordinate form} as
\label{prob:tri_polar}
%
\begin{align}
\vec{A} = \myvec{b\cos \theta\\ b \sin \theta} 
\end{align}
%
\fi

\end{enumerate}


\section{Sum of Angles}
\input{./chapters/baudh/tri_geo_sum_angle}
\section{Proof of Baudhayana Theorem}
\input{./chapters/baudh/tri_geo_baudh}


\chapter{Area of a Triangle}
\section{From a Rectangle}
\input{./chapters/area/tri_geo_rect_area}
\section{Sine and Cosine formula}
\iffalse
\begin{figure}[!ht]
\centering
\resizebox{\columnwidth}{!}{\input{./figs/triangle/tri_right_angle.tex}}
\caption{}
\label{fig:tri_right_angle_area}	
\end{figure}
\fi
\renewcommand{\theequation}{\theenumi}
\begin{enumerate}[label=\thesection.\arabic*.,ref=\thesection.\theenumi]
\numberwithin{equation}{enumi}


\item
	Show that 
	\begin{equation}
	\frac{\sin A}{a} = \frac{\sin B}{b} = \frac{\sin C}{c}
	\end{equation}

\solution Fig. \ref{fig:tri_sss} can be suitably modified to obtain 
\begin{align}
ar\brak{\Delta ABC} = 
\frac{1}{2}ab\sin C = \frac{1}{2}bc\sin A = \frac{1}{2}ca\sin B
\end{align}
Dividing the above by $abc$, we obtain
	\begin{equation}
\label{eq:tri_sin_form}
	\frac{\sin A}{a} = \frac{\sin B}{b} = \frac{\sin C}{c}
	\end{equation}
This is known as the sine formula.	
%
%
\item Show that 
%
\begin{align}
\label{eq:trig_id_sin_inc}
\alpha > \beta \implies \sin \alpha > \sin \beta
\end{align}
%

\begin{figure}[!ht]
	\begin{center}
		
		%\includegraphics[width=\columnwidth]{./figs/fig:tri_sin_inc}
		%\vspace*{-10cm}
		\resizebox{\columnwidth}{!}{\input{./figs/triangle/tri_sin_inc.tex}}
	\end{center}
	\caption{}
	%\caption{$\sin \brak{\theta_1+\theta_2} = \sin\theta_1\cos\theta_2 + \cos\theta_1\sin\theta_2$}
	\label{fig:tri_sin_inc}	
\end{figure}
\solution In Fig. \ref{fig:tri_sin_inc}, 	
%
\begin{align}
ar\brak{\triangle ABD} &< ar \brak{\triangle ABC}
\\
\implies \frac{1}{2}lc \sin \theta_1 &<  \frac{1}{2}ac \sin \brak{\theta_1 + \theta_2 }
\\
\implies \frac{l}{a} &< \frac{\sin \brak{\theta_1 + \theta_2 }}{\sin \theta_1}
\\
\text{or, } 1 < \frac{l}{a} &< \frac{\sin \brak{\theta_1 + \theta_2 }}{\sin \theta_1}
\\
\implies \frac{\sin \brak{\theta_1 + \theta_2 }}{\sin \theta_1} > 1
\end{align}
%
from Theorem \ref{them:hyp_largest}. This proves \eqref{eq:trig_id_sin_inc}.
%From \eqref{eq:trig_id_sum_diff3},
%%
%\begin{multline}
% \sin \theta_1 - \sin \theta_2 = 2\sin\brak{\frac{\theta_1-\theta_2}{2}}
%\\
%\times \cos\brak{\frac{\theta_1+\theta_2}{2}} > 0, \because \theta_1-\theta_2 > 0
%\end{multline}
%
\item In a triangle, the side opposite the greater angle is greater.
\begin{figure}[!ht]
	\begin{center}
			\resizebox{\columnwidth}{!}{\input{./figs/quad/tri_ang_side.tex}}
	\end{center}
	\caption{Side opposite the greater angle is greater}
	\label{fig:tri_ang_side}	
\end{figure}
\\
\solution In Fig. 	\ref{fig:tri_ang_side},	let
%
\begin{align}
\angle B > \angle C
\end{align}
%
Then, using the sine formula,
%
\begin{align}
\frac{\sin B}{b} &=\frac{\sin C}{c}
\\
\implies   \frac{\sin B}{\sin C} &= \frac{b}{c} > 1
\end{align}
using \eqref{eq:trig_id_sin_inc}.


\end{enumerate}

\section{Hero's formula}
\input{./chapters/area/tri_geo_hero}

\chapter{Circumcentre}
\section{Locating the Circumcentre}
\input{./chapters/circumc/tri_geo_ccentre}
\section{Finding the Circumradius}
\input{./chapters/circumc/tri_geo_cradius}
\section{Drawing the Circumcircle}
\input{./chapters/circumc/tri_geo_ccircle}

\chapter{Incentre}
\section{Locating the Incentre}
\input{./chapters/inc/tri_geo_icentre}
\section{Drawing the Incircle}
\input{./chapters/inc/tri_geo_icircle}
\section{Congruent Triangles}
\input{./chapters/inc/tri_geo_cong}

\chapter{Medians of a Triangle}
\section{Basic Proportionality Theorem}
\input{./chapters/median/tri_geo_bpt}
\section{Similar Triangles}
\input{./chapters/median/tri_geo_sim}
\section{Centroid}
\input{./chapters/median/tri_geo_centroid}
\section{Median and Area}
\input{./chapters/median/tri_geo_med_area}
\section{Parallelogram}
\input{./chapters/quad/quad_geo_prop}
%

\chapter{Circles}
\section{Area of a Circle}
\input{./chapters/area/circ_geo_area}
\section{Miscellaneous Properties}
\input{./chapters/misc/circ_geo_prop}

\chapter{Miscellaneous}
\section{Trigonometric Identities}
\input{./chapters/quad/trig_id}
\iffalse
\section{Constructions}
\input{./chapters/quad/quad_const_exam}
\section{Altitudes of a Triangle}
\input{./chapters/misc/tri_geo_alt}
\chapter{ Vectors}
\section{$2\times 1$ vectors}
\input{matrix/two.tex}
%\include{app01}
%\appendix
\section{$3\times 1$ vectors}
\input{matrix/three.tex}
\chapter{Matrices}
\input{matrix/mat.tex}

\chapter{Linear Forms}
\section{Two Dimensions}
\input{linear/two.tex}
\section{Three Dimensions}
\input{linear/three.tex}
\chapter{Quadratic Forms}
%\numberwithin{equation}{section}
%\numberwithin{equation}{section}
\section{Conic equation }
\input{quad/defs.tex}
\section{Circles}
\input{quad/circle.tex}

\section{Standard Form}
\input{quad/stddef.tex}
\chapter{Conic Parameters}
\section{Standard Form}
\input{quad/standard.tex}
\section{Quadratic Form }
\input{quad/coroll.tex}

\chapter{Conic Lines}
\section{Pair of Straight Lines}
%
\input{quad/pair.tex}
\section{Intersection of Conics}
\input{quadlines/inter.tex}
\section{ Chords of a Conic}
\input{quadlines/chord.tex}
\section{ Tangent and Normal}
%%
%\subsection{Perpendicular Bisectors}
\renewcommand{\theequation}{\theenumi}
\begin{enumerate}[label=\thesection.\arabic*.,ref=\thesection.\theenumi]
\numberwithin{equation}{enumi}
\item
	In Fig. \ref{fig:circ_tang_icept}, show that $PA.PB = PC^2$.
\label{them:circ_tang_icept_prod}	
\\
\solution 
In $\triangle$s $APC$ and $BPC$, 
using
		\eqref{fig:circ_tang_icept-equal},	
  \begin{align}
	  \frac{AP}{\sin \theta} &= \frac{AC}{\sin P} 
	  \\
	  \frac{PC}{\sin \theta} &= \frac{BC}{\sin P} 
	  \\
	  \implies \frac{PC}{AP} &= \frac{BC}{AC}  \brak{= \frac{BP}{CP}}
  \end{align}
  which gives the desired result.
$\triangle$s $APC$ and $BPC$ are said to be {\em similar}.
\end{enumerate}

\fi
%\chapter{Proofs}
%   \section{}
%\input{apps/defs.tex}

%  \section{}
%\input{apps/parab.tex}
%  \section{}
%\input{apps/nonparab.tex}
%		\section{}
%\input{apps/params.tex}
\latexprintindex

\end{document}

 

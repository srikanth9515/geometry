%\renewcommand{\theequation}{\theenumi}
%\begin{enumerate}[label=\arabic*.,ref=\theenumi]
\begin{enumerate}[label=\thesection.\arabic*.,ref=\thesection.\theenumi]
\numberwithin{equation}{enumi}
\item Suppose the equations $AB, BC$ and $CA$ are respectively given by 
		\begin{align}
			\label{eq:tri-sides}
			\vec{n}_i^{\top}\vec{x}=c_i \quad i = 1, 2, 3 
		\end{align}
		The equations of the respective angle bisectors are then given by 
		\begin{align}
			\frac{\vec{n}_i^{\top}\vec{x}-c_i}{\norm{\vec{n}_i}}
		=
	\pm	\frac{\vec{n}_j^{\top}\vec{x}-c_j}{\norm{\vec{n}_j}}
\quad i \ne j
		\end{align}
		Substitute numerical values and find the equations of the angle bisectors of $A, B$ and $C$.
	\\
		%\textbf{Solution :}
	The parametric equations of sides;
	\begin{align}
	BC:\quad &\myvec{11&1}\vec{x}=-38,\\
	CA:\quad &\myvec{1&-1}\vec{x}=2,\\
	AB:\quad &\myvec{7&5}\vec{x}=2\\	  
	\end{align}
	Using the formula mentioned in the question to find out the angular bisector for sides \text{AB} and \text{AC}, naming the angular bisector $L$ we get,
	\begin{align}
		\frac{\vec{n}_{3}^{\top} \vec{x}-c_{3}}{\norm{\vec{n}_{3}}}=\pm \frac{\vec{n}_{2}^{\top} \vec{x}-c_{2}}{\norm{\vec{n}_{2}}}
	\end{align}
	\begin{figure}
	\centering
	\includegraphics[width=\columnwidth]{solutions/1/5/1/figs/angular_bisector.png}
	\caption{Triangle generated using python}
	\label{fig:angular_bisector}
	\end{figure}
	As we can see we will get 2 solutions for $L$. This is because one of them is internal angular bisector and the other is the external angular bisector. Internal angular bisector can be evaluated if we take + in the above formula.
	Hence, $L$ is given by,
	\begin{align}
		\frac{\vec{n}_{3}^{\top} \vec{x}-c_{3}}{\norm{\vec{n}_{3}}}&=\frac{\vec{n}_{2}^{\top} \vec{x}-c_{2}}{\norm{\vec{n}_{2}}}\\
		\implies \brak{\frac{\vec{n_{3}}}{\norm{\vec{n_{3}}}}-\frac{\vec{n_{3}}}{\norm{\vec{n_{3}}}}} \vec{x}&=\brak{\frac{c_{3}}{\norm{\vec{n_{3}}}}-\frac{c_{2}}{\norm{\vec{n_{2}}}}}\\
		\implies \brak{\frac{\myvec{7&5}}{\sqrt{74}}-\frac{\myvec{1&-1}}{\sqrt{2}}} \vec{x}&=\frac{2}{\sqrt{74}}-\frac{2}{\sqrt{2}}\\
		\implies \myvec{\frac{7-\sqrt{37}}{\sqrt{74}}&\frac{5+\sqrt{37}}{\sqrt{74}}}\vec{x}&=\frac{2-2\sqrt{37}}{\sqrt{74}}
	\end{align}
	Hence, the internal angluar bisector of angle $A$, $L$ will be,
	\begin{align}
		\implies\myvec{\frac{7-\sqrt{37}}{\sqrt{74}}&\frac{5+\sqrt{37}}{\sqrt{74}}} \vec{x}=\frac{2-2\sqrt{37}}{\sqrt{74}}
		\label{eq:1.5.1}
	\end{align}
	

  \iffalse
\documentclass[journal,12pt,twocolumn]{IEEEtran}
\usepackage{cite}
\usepackage{amsmath,amssymb,amsfonts,amsthm}
\usepackage{algorithmic}
\usepackage{graphicx}
\usepackage{textcomp}
\usepackage{xcolor}
\usepackage{txfonts}
\usepackage{listings}
\usepackage{enumitem}
\usepackage{mathtools}
\usepackage{gensymb}
\usepackage[breaklinks=true]{hyperref}
\usepackage{tkz-euclide} % loads  TikZ and tkz-base
\usepackage{listings}
\usepackage{float}


\begin{document}
\providecommand{\pr}[1]{\ensuremath{\Pr\left(#1\right)}}
\providecommand{\prt}[2]{\ensuremath{p_{#1}^{\left(#2\right)} }}        % own macro for this question
\providecommand{\qfunc}[1]{\ensuremath{Q\left(#1\right)}}
\providecommand{\sbrak}[1]{\ensuremath{{}\left[#1\right]}}
\providecommand{\lsbrak}[1]{\ensuremath{{}\left[#1\right.}}
\providecommand{\rsbrak}[1]{\ensuremath{{}\left.#1\right]}}
\providecommand{\brak}[1]{\ensuremath{\left(#1\right)}}
\providecommand{\lbrak}[1]{\ensuremath{\left(#1\right.}}
\providecommand{\rbrak}[1]{\ensuremath{\left.#1\right)}}
\providecommand{\cbrak}[1]{\ensuremath{\left\{#1\right\}}}
\providecommand{\lcbrak}[1]{\ensuremath{\left\{#1\right.}}
\providecommand{\rcbrak}[1]{\ensuremath{\left.#1\right\}}}
\newcommand{\sgn}{\mathop{\mathrm{sgn}}}
\providecommand{\abs}[1]{\left\vert#1\right\vert}
\providecommand{\res}[1]{\Res\displaylimits_{#1}} 
\providecommand{\norm}[1]{\left\lVert#1\right\rVert}
%\providecommand{\norm}[1]{\lVert#1\rVert}
\providecommand{\mtx}[1]{\mathbf{#1}}
\providecommand{\mean}[1]{E\left[ #1 \right]}
\providecommand{\cond}[2]{#1\middle|#2}
\providecommand{\fourier}{\overset{\mathcal{F}}{ \rightleftharpoons}}
\newenvironment{amatrix}[1]{%
  \left(\begin{array}{@{}*{#1}{c}|c@{}}
}{%
  \end{array}\right)
}
%\providecommand{\hilbert}{\overset{\mathcal{H}}{ \rightleftharpoons}}
%\providecommand{\system}{\overset{\mathcal{H}}{ \longleftrightarrow}}
	%\newcommand{\solution}[2]{\textbf{Solution:}{#1}}
\newcommand{\solution}{\noindent \textbf{Solution: }}
\newcommand{\cosec}{\,\text{cosec}\,}
\providecommand{\dec}[2]{\ensuremath{\overset{#1}{\underset{#2}{\gtrless}}}}
\newcommand{\myvec}[1]{\ensuremath{\begin{pmatrix}#1\end{pmatrix}}}
\newcommand{\mydet}[1]{\ensuremath{\begin{vmatrix}#1\end{vmatrix}}}
\newcommand{\myaugvec}[2]{\ensuremath{\begin{amatrix}{#1}#2\end{amatrix}}}
\providecommand{\rank}{\text{rank}}
\providecommand{\pr}[1]{\ensuremath{\Pr\left(#1\right)}}
\providecommand{\qfunc}[1]{\ensuremath{Q\left(#1\right)}}
	\newcommand*{\permcomb}[4][0mu]{{{}^{#3}\mkern#1#2_{#4}}}
\newcommand*{\perm}[1][-3mu]{\permcomb[#1]{P}}
\newcommand*{\comb}[1][-1mu]{\permcomb[#1]{C}}
\providecommand{\qfunc}[1]{\ensuremath{Q\left(#1\right)}}
\providecommand{\gauss}[2]{\mathcal{N}\ensuremath{\left(#1,#2\right)}}
\providecommand{\diff}[2]{\ensuremath{\frac{d{#1}}{d{#2}}}}
\providecommand{\myceil}[1]{\left \lceil #1 \right \rceil }
\newcommand\figref{Fig.~\ref}
\newcommand\tabref{Table~\ref}
\newcommand{\sinc}{\,\text{sinc}\,}
\newcommand{\rect}{\,\text{rect}\,}
%%
%	%\newcommand{\solution}[2]{\textbf{Solution:}{#1}}
%\newcommand{\solution}{\noindent \textbf{Solution: }}
%\newcommand{\cosec}{\,\text{cosec}\,}
%\numberwithin{equation}{section}
%\numberwithin{equation}{subsection}
%\numberwithin{problem}{section}
%\numberwithin{definition}{section}
%\makeatletter
%\@addtoreset{figure}{problem}
%\makeatother

%\let\StandardTheFigure\thefigure
\let\vec\mathbf

\bibliographystyle{IEEEtran}


\vspace{3cm}

\textbf{Question 1.5.1}\\
Suppose the equations $AB, BC$ and $CA$ are respectively given by 
		\begin{align}
			\label{eq:tri-sides}
			\vec{n}_i^{\top}\vec{x}=c_i \quad i = 1, 2, 3 
		\end{align}
		The equations of the respective angle bisectors are then given by 
		\begin{align}
			\frac{\vec{n}_i^{\top}\vec{x}-c_i}{\norm{\vec{n}_i}}
		=
	\pm	\frac{\vec{n}_j^{\top}\vec{x}-c_j}{\norm{\vec{n}_j}}
\quad i \ne j
		\end{align}
		Substitute numerical values and find the equations of the angle bisectors of $A, B$ and $C$.\\
\fi
\solution 
The internal angle bisector is obtained from the set of two bisectors by using:
		\begin{align}
			\frac{\vec{n}_i^{\top}\vec{x}-c_i}{\norm{\vec{n}_i}}
		=
		\frac{\vec{n}_j^{\top}\vec{x}-c_j}{\norm{\vec{n}_j}}
\quad i \ne j	
\end{align}
This can be transformed to the normal equation of angle bisectors as follows 
\begin{align}
       \myvec{\frac{\vec{n}_i^{\top}}{\norm{\vec{n}_i}} - \frac{\vec{n}_j^{\top}}{\norm{\vec{n}_j}}}\vec{x}
       =
       \frac{c_i}{\norm{\vec{n}_i}}-\frac{c_j}{\norm{\vec{n}_j}}
\end{align}
$i$ and $j$ values correspond to the sides including the angle\\
\begin{enumerate}
\item Angle Bisector of $A$
\begin{align}
       \myvec{\frac{\vec{n}_3^{\top}}{\norm{\vec{n}_3}} - \frac{\vec{n}_1^{\top}}{\norm{\vec{n}_1}}}\vec{x}
       =
       \frac{c_3}{\norm{\vec{n}_3}}-\frac{c_1}{\norm{\vec{n}_1}}
\end{align}
on substitution we obtain 
\begin{align}
\myvec{
\frac{7}{\sqrt{74}}-\frac{1}{\sqrt{2}} & \frac{5}{\sqrt{74}}+\frac{1}{\sqrt{2}}\\
}
\vec{x}
=\frac{2}{\sqrt{74}}-\frac{2}{\sqrt{2}}
\end{align}
\item Angle Bisector of $B$
\begin{align}
       \myvec{\frac{\vec{n}_2^{\top}}{\norm{\vec{n}_2}} - \frac{\vec{n}_1^{\top}}{\norm{\vec{n}_1}}}\vec{x}
       =
       \frac{c_2}{\norm{\vec{n}_2}}-\frac{c_1}{\norm{\vec{n}_1}}
\end{align}
on substitution we obtain 
\begin{align}
\myvec{
\frac{11}{\sqrt{122}}+\frac{7}{\sqrt{74}} & \frac{1}{\sqrt{122}}+\frac{5}{\sqrt{74}}\\
}
\vec{x}
=\frac{2}{\sqrt{74}}-\frac{38}{\sqrt{122}}
\end{align}
\item Angle Bisector of $C$
\begin{align}
       \myvec{\frac{\vec{n}_2^{\top}}{\norm{\vec{n}_2}} - \frac{\vec{n}_3^{\top}}{\norm{\vec{n}_3}}}\vec{x}
       =
       \frac{c_2}{\norm{\vec{n}_2}}-\frac{c_3}{\norm{\vec{n}_3}}
\end{align}
on substitution we obtain 
\begin{align}
\myvec{
\frac{11}{\sqrt{122}}+\frac{1}{\sqrt{2}} & \frac{1}{\sqrt{122}}-\frac{1}{\sqrt{2}}\\
}
\vec{x}
=\frac{2}{\sqrt{2}}-\frac{38}{\sqrt{122}}
\end{align}
\end{enumerate}
\begin{figure}[H]
\includegraphics[width=\columnwidth]{solutions/1/5/1(1)/figs/anglebisector.png}
\caption{Angle bisectors plotted using python}
\label{fig:i_angbisector_py}
\end{figure}

	\item Find the intersection $\vec{I}$ of the angle bisectors of $B$ and $C$.
 \\
		\iffalse
\documentclass[journal,12pt,twocolumn]{IEEEtran}
\usepackage{cite}
\usepackage{amsmath,amssymb,amsfonts,amsthm}
\usepackage{algorithmic}
\usepackage{graphicx}
\usepackage{textcomp}
\usepackage{xcolor}
\usepackage{txfonts}
\usepackage{listings}
\usepackage{enumitem}
\usepackage{mathtools}
\usepackage{gensymb}
\usepackage[breaklinks=true]{hyperref}
\usepackage{tkz-euclide} % loads  TikZ and tkz-base
\usepackage{listings}
\usepackage{float}


\begin{document}
\providecommand{\pr}[1]{\ensuremath{\Pr\left(#1\right)}}
\providecommand{\prt}[2]{\ensuremath{p_{#1}^{\left(#2\right)} }}        % own macro for this question
\providecommand{\qfunc}[1]{\ensuremath{Q\left(#1\right)}}
\providecommand{\sbrak}[1]{\ensuremath{{}\left[#1\right]}}
\providecommand{\lsbrak}[1]{\ensuremath{{}\left[#1\right.}}
\providecommand{\rsbrak}[1]{\ensuremath{{}\left.#1\right]}}
\providecommand{\brak}[1]{\ensuremath{\left(#1\right)}}
\providecommand{\lbrak}[1]{\ensuremath{\left(#1\right.}}
\providecommand{\rbrak}[1]{\ensuremath{\left.#1\right)}}
\providecommand{\cbrak}[1]{\ensuremath{\left\{#1\right\}}}
\providecommand{\lcbrak}[1]{\ensuremath{\left\{#1\right.}}
\providecommand{\rcbrak}[1]{\ensuremath{\left.#1\right\}}}
\newcommand{\sgn}{\mathop{\mathrm{sgn}}}
\providecommand{\abs}[1]{\left\vert#1\right\vert}
\providecommand{\res}[1]{\Res\displaylimits_{#1}} 
\providecommand{\norm}[1]{\left\lVert#1\right\rVert}
%\providecommand{\norm}[1]{\lVert#1\rVert}
\providecommand{\mtx}[1]{\mathbf{#1}}
\providecommand{\mean}[1]{E\left[ #1 \right]}
\providecommand{\cond}[2]{#1\middle|#2}
\providecommand{\fourier}{\overset{\mathcal{F}}{ \rightleftharpoons}}
\newenvironment{amatrix}[1]{%
  \left(\begin{array}{@{}*{#1}{c}|c@{}}
}{%
  \end{array}\right)
}
%\providecommand{\hilbert}{\overset{\mathcal{H}}{ \rightleftharpoons}}
%\providecommand{\system}{\overset{\mathcal{H}}{ \longleftrightarrow}}
	%\newcommand{\solution}[2]{\textbf{Solution:}{#1}}
\newcommand{\solution}{\noindent \textbf{Solution: }}
\newcommand{\cosec}{\,\text{cosec}\,}
\providecommand{\dec}[2]{\ensuremath{\overset{#1}{\underset{#2}{\gtrless}}}}
\newcommand{\myvec}[1]{\ensuremath{\begin{pmatrix}#1\end{pmatrix}}}
\newcommand{\mydet}[1]{\ensuremath{\begin{vmatrix}#1\end{vmatrix}}}
\newcommand{\myaugvec}[2]{\ensuremath{\begin{amatrix}{#1}#2\end{amatrix}}}
\providecommand{\rank}{\text{rank}}
\providecommand{\pr}[1]{\ensuremath{\Pr\left(#1\right)}}
\providecommand{\qfunc}[1]{\ensuremath{Q\left(#1\right)}}
	\newcommand*{\permcomb}[4][0mu]{{{}^{#3}\mkern#1#2_{#4}}}
\newcommand*{\perm}[1][-3mu]{\permcomb[#1]{P}}
\newcommand*{\comb}[1][-1mu]{\permcomb[#1]{C}}
\providecommand{\qfunc}[1]{\ensuremath{Q\left(#1\right)}}
\providecommand{\gauss}[2]{\mathcal{N}\ensuremath{\left(#1,#2\right)}}
\providecommand{\diff}[2]{\ensuremath{\frac{d{#1}}{d{#2}}}}
\providecommand{\myceil}[1]{\left \lceil #1 \right \rceil }
\newcommand\figref{Fig.~\ref}
\newcommand\tabref{Table~\ref}
\newcommand{\sinc}{\,\text{sinc}\,}
\newcommand{\rect}{\,\text{rect}\,}
%%
%	%\newcommand{\solution}[2]{\textbf{Solution:}{#1}}
%\newcommand{\solution}{\noindent \textbf{Solution: }}
%\newcommand{\cosec}{\,\text{cosec}\,}
%\numberwithin{equation}{section}
%\numberwithin{equation}{subsection}
%\numberwithin{problem}{section}
%\numberwithin{definition}{section}
%\makeatletter
%\@addtoreset{figure}{problem}
%\makeatother

%\let\StandardTheFigure\thefigure
\let\vec\mathbf

\bibliographystyle{IEEEtran}


\vspace{3cm}

\textbf{Question 1.5.2}\\
Find the intersection $\vec{I}$ of the angle bisectors of $\vec{B}$ and $\vec{C}$
\fi
\textbf{Solution}:\\
From 1.5.1 the bisectors of $\vec{B}$ and $\vec{C}$ are obtained as
\begin{align}
\myvec{
\frac{11}{\sqrt{122}}+\frac{7}{\sqrt{74}} & \frac{1}{\sqrt{122}}+\frac{5}{\sqrt{74}}\\
}
\vec{x}
=\frac{2}{\sqrt{74}}-\frac{38}{\sqrt{122}}
\end{align}
and 
\begin{align}
\myvec{
\frac{11}{\sqrt{122}}+\frac{1}{\sqrt{2}} & \frac{1}{\sqrt{122}}-\frac{1}{\sqrt{2}}\\
}
\vec{x}
=\frac{2}{\sqrt{2}}-\frac{38}{\sqrt{122}}
\end{align}
respectively.
The pair of linear equations can be solved using the Augmented matrix $\myvec{
\vec{P}|\vec{Q}}$
Here,
\begin{align}
\vec{P}&=\myvec{
\frac{11}{\sqrt{122}}+\frac{7}{\sqrt{74}} & \frac{1}{\sqrt{122}}+\frac{5}{\sqrt{74}}\\
\frac{11}{\sqrt{122}}+\frac{1}{\sqrt{2}} & \frac{1}{\sqrt{122}}-\frac{1}{\sqrt{2}}\\
}\\
\vec{Q}&=\myvec{
\frac{2}{\sqrt{74}}-\frac{38}{\sqrt{122}}\
\frac{2}{\sqrt{2}}-\frac{38}{\sqrt{122}}\\
}\\
\myaugvec{1}{\vec{P}&\vec{Q}} 
 &= \myaugvec{2}
 {
\frac{11}{\sqrt{122}}+\frac{7}{\sqrt{74}} & \frac{1}{\sqrt{122}}+\frac{5}{\sqrt{74}} & \frac{2}{\sqrt{74}}-\frac{38}{\sqrt{122}} \\
\frac{11}{\sqrt{122}}+\frac{1}{\sqrt{2}} & \frac{1}{\sqrt{122}}-\frac{1}{\sqrt{2}} & \frac{2}{\sqrt{2}}-\frac{38}{\sqrt{122}} \\
}
\end{align}
The augmented matrix is converted into decimal notations for easier calculations and then can be solved using row reduction as follows 
\begin{align}
\myaugvec{2}
{
1.81 & 0.67 & -3.21 \\
 1.7 & -0.62 & -2.03\\
}
 \xleftrightarrow[]{R_2\leftarrow 1.7R_1-1.81R_2}
\myaugvec{2}
{1.81 & 0.67 & -3.21 \\
0 & 1.33 & -1.05\\
}\\
\xleftrightarrow[]{R_1\leftarrow 1.33R_1-0.67R_2}
\myaugvec{2}
{1.81 & 0 & -2.68 \\
0 & 1.33 & -1.05\\
}\\
\xleftrightarrow[]{R_1\leftarrow \frac{R_1}{1.81}}
\myaugvec{2}
{
1 & 0 & -1.48 \\
0 & 1.33 & -1.05\\
}\\
\xleftrightarrow[]{R_2\leftarrow \frac{R_2}{1.33}}
\myaugvec{2}{1 & 0 & -1.48 \\ 0 & 1 & -0.79} 
\end{align}
We obtain 
\begin{align}
\vec{I}=\myvec{-1.48\\-0.79}
\end{align}
\begin{figure}[H]
\includegraphics[width=\columnwidth]{solutions/1/5/2/figs/Incentre.png}
\caption{Intersection point $\vec{I}$ of angle bisectors of $\vec{B}$and$\vec{C}$ plotted using python}
\label{fig:i_tri_py}
\end{figure}



	\item Using 
    \eqref{eq:angle2d}
verify that 
		\begin{align}
			\angle BAI = \angle CAI.
		\end{align}
	\\	\iffalse
\let\negmedspace\undefined
\let\negthickspace\undefined
\documentclass[journal,12pt,twocolumn]{IEEEtran}
\usepackage{cite}
\usepackage{amsmath,amssymb,amsfonts,amsthm}
\usepackage{algorithmic}
\usepackage{graphicx}
\usepackage{textcomp}
\usepackage{xcolor}
\usepackage{txfonts}
\usepackage{listings}
\usepackage{enumitem}
\usepackage{mathtools}
\usepackage{gensymb}
\usepackage[breaklinks=true]{hyperref}
\usepackage{tkz-euclide} % loads  TikZ and tkz-base
\usepackage{listings}
\usepackage{gvv}
%
%\usepackage{setspace}
%\usepackage{gensymb}
%\doublespacing
%\singlespacing

%\usepackage{graphicx}
%\usepackage{amssymb}
%\usepackage{relsize}
%\usepackage[cmex10]{amsmath}
%\usepackage{amsthm}
%\interdisplaylinepenalty=2500
%\savesymbol{iint}
%\usepackage{txfonts}
%\restoresymbol{TXF}{iint}
%\usepackage{wasysym}
%\usepackage{amsthm}
%\usepackage{iithtlc}
%\usepackage{mathrsfs}
%\usepackage{txfonts}
%\usepackage{stfloats}
%\usepackage{bm}
%\usepackage{cite}
%\usepackage{cases}
%\usepackage{subfig}
%\usepackage{xtab}
%\usepackage{longtable}
%\usepackage{multirow}
%\usepackage{algorithm}
%\usepackage{algpseudocode}
%\usepackage{enumitem}
%\usepackage{mathtools}
%\usepackage{tikz}
%\usepackage{circuitikz}
%\usepackage{verbatim}
%\usepackage{tfrupee}
%\usepackage{stmaryrd}
%\usetkzobj{all}
%    \usepackage{color}                                            %%
%    \usepackage{array}                                            %%
%    \usepackage{longtable}                                        %%
%    \usepackage{calc}                                             %%
%    \usepackage{multirow}                                         %%
%    \usepackage{hhline}                                           %%
%    \usepackage{ifthen}                                           %%
  %optionally (for landscape tables embedded in another document): %%
%    \usepackage{lscape}     
%\usepackage{multicol}
%\usepackage{chngcntr}
%\usepackage{enumerate}

%\usepackage{wasysym}
%\documentclass[conference]{IEEEtran}
%\IEEEoverridecommandlockouts
% The preceding line is only needed to identify funding in the first footnote. If that is unneeded, please comment it out.

\newtheorem{theorem}{Theorem}[section]
\newtheorem{problem}{Problem}
\newtheorem{proposition}{Proposition}[section]
\newtheorem{lemma}{Lemma}[section]
\newtheorem{corollary}[theorem]{Corollary}
\newtheorem{example}{Example}[section]
\newtheorem{definition}[problem]{Definition}
%\newtheorem{thm}{Theorem}[section] 
%\newtheorem{defn}[thm]{Definition}
%\newtheorem{algorithm}{Algorithm}[section]
%\newtheorem{cor}{Corollary}
\newcommand{\BEQA}{\begin{eqnarray}}
\newcommand{\EEQA}{\end{eqnarray}}
\newcommand{\define}{\stackrel{\triangle}{=}}
\theoremstyle{remark}
\newtheorem{rem}{Remark}

%\bibliographystyle{ieeetr}
\setlength{\parindent}{0pt}
\begin{document}
\bibliographystyle{IEEEtran}


\vspace{3cm}

\title{
%	\logo{
EE23010 Assignment
%	}
}
\author{ Sayyam Palrecha$^{*}$ EE22BTECH11047% <-this % stops a space
	\thanks{}
	
}
%\title{
%	\logo{Matrix Analysis through Octave}{\begin{center}\includegraphics[scale=.24]{tlc}\end{center}}{}{HAMDSP}
%}


% paper title
% can use linebreaks \\ within to get better formatting as desired
%\title{Matrix Analysis through Octave}
%
%
% author names and IEEE memberships
% note positions of commas and nonbreaking spaces ( ~ ) LaTeX will not break
% a structure at a ~ so this keeps an author's name from being broken across
% two lines.
% use \thanks{} to gain access to the first footnote area
% a separate \thanks must be used for each paragraph as LaTeX2e's \thanks
% was not built to handle multiple paragraphs
%

%\author{<-this % stops a space
%\thanks{}}
%}
% note the % following the last \IEEEmembership and also \thanks - 
% these prevent an unwanted space from occurring between the last author name
% and the end of the author line. i.e., if you had this:
% 
% \author{....lastname \thanks{...} \thanks{...} }
%                     ^------------^------------^----Do not want these spaces!
%
% a space would be appended to the last name and could cause every name on that
% line to be shifted left slightly. This is one of those "LaTeX things". For
% instance, "\textbf{A} \textbf{B}" will typeset as "A B" not "AB". To get
% "AB" then you have to do: "\textbf{A}\textbf{B}"
% \thanks is no different in this regard, so shield the last } of each \thanks
% that ends a line with a % and do not let a space in before the next \thanks.
% Spaces after \IEEEmembership other than the last one are OK (and needed) as
% you are supposed to have spaces between the names. For what it is worth,
% this is a minor point as most people would not even notice if the said evil
% space somehow managed to creep in.



% The paper headers
%\markboth{Journal of \LaTeX\ Class Files,~Vol.~6, No.~1, January~2007}%
%{Shell \MakeLowercase{\textit{et al.}}: Bare Demo of IEEEtran.cls for Journals}
% The only time the second header will appear is for the odd numbered pages
% after the title page when using the twoside option.
% 
% *** Note that you probably will NOT want to include the author's ***
% *** name in the headers of peer review papers.                   ***
% You can use \ifCLASSOPTIONpeerreview for conditional compilation here if
% you desire.




% If you want to put a publisher's ID mark on the page you can do it like
% this:
%\IEEEpubid{0000--0000/00\$00.00~\copyright~2007 IEEE}
% Remember, if you use this you must call \IEEEpubidadjcol in the second
% column for its text to clear the IEEEpubid mark.



% make the title area
\maketitle

\newpage

%\tableofcontents

\bigskip

\renewcommand{\thefigure}{\theenumi}
\renewcommand{\thetable}{\theenumi}
%\renewcommand{\theequation}{\theenumi}

%\begin{abstract}
%%\boldmath
%In this letter, an algorithm for evaluating the exact analytical bit error rate  (BER)  for the piecewise linear (PL) combiner for  multiple relays is presented. Previous results were available only for upto three relays. The algorithm is unique in the sense that  the actual mathematical expressions, that are prohibitively large, need not be explicitly obtained. The diversity gain due to multiple relays is shown through plots of the analytical BER, well supported by simulations. 
%
%\end{abstract}
% IEEEtran.cls defaults to using nonbold math in the Abstract.
% This preserves the distinction between vectors and scalars. However,
% if the journal you are submitting to favors bold math in the abstract,
% then you can use LaTeX's standard command \boldmath at the very start
% of the abstract to achieve this. Many IEEE journals frown on math
% in the abstract anyway.

% Note that keywords are not normally used for peerreview papers.
%\begin{IEEEkeywords}
%Cooperative diversity, decode and forward, piecewise linear
%\end{IEEEkeywords}



% For peer review papers, you can put extra information on the cover
% page as needed:
% \ifCLASSOPTIONpeerreview
% \begin{center} \bfseries EDICS Category: 3-BBND \end{center}
% \fi
%
% For peerreview papers, this IEEEtran command inserts a page break and
% creates the second title. It will be ignored for other modes.
%\IEEEpeerreviewmaketitle


Question 1.5.3
Using (1.1.7.1) verify that 
\begin{align}\angle BAI = \angle CAI.\end{align}
\fi
Given:
\begin{align}
\vec{A} = \myvec{1\\-1}\\
\vec{B} = \myvec{-4\\6}\\
\vec{C} = \myvec{-3\\-5}
\end{align}
The intersection $\vec{I}$ of the angle bisectors of $B$ and $C$:
\begin{align}
\vec{I} &= \frac{1}{\sqrt{37}+ 4 +\sqrt{61}}\myvec{
{\sqrt{61}-16-3\sqrt{37}}\\
{-\sqrt{61}+24-5\sqrt{37}}
}\\
\cos{A} &= \frac{\brak{\vec{B} - \vec{A}}^{\top}\vec{C} - \vec{A}}{\norm{\vec{B} - \vec{A}} \norm{ \vec{C} - \vec{A}}}
\end{align}
\begin{figure}
\centering
\includegraphics[width=\columnwidth]{solutions/1/5/3/fig/main.png}
\caption{Triangle generated using python}
\label{fig:tri_sss_py}
\end{figure}
\solution

We need to verify \begin{align}\angle BAI &= \angle CAI.\end{align}
consider LHS:
\begin{align} 
\cos{\angle BAI} &= \frac{\brak{\vec{B} - \vec{A}}^{\top}\vec{I} - \vec{A}}{\norm{\vec{B} - \vec{A}} \norm{ \vec{I} - \vec{A}}}\\
 &= \frac{\brak{\myvec{-4\\6} - \myvec{1\\-1}}^{\top}\myvec{\frac{{\sqrt{61}-16-3\sqrt{37}}}{\sqrt{37}+4+\sqrt{61}}\\\frac{{-\sqrt{61}+24-5\sqrt{37}}}{\sqrt{37}+4+\sqrt{61}}} - \myvec{1\\-1}}{\norm{\myvec{-4\\6} - \myvec{1\\-1}} \norm{ \myvec{\frac{{\sqrt{61}-16-3\sqrt{37}}}{\sqrt{37}+4+\sqrt{61}}\\\frac{{-\sqrt{61}+24-5\sqrt{37}}}{\sqrt{37}+4+\sqrt{61}}} - \myvec{1\\-1}}}\\
 &= \frac{\myvec{-5\\7}^{\top}\myvec{\frac{{\sqrt{61}-16-3\sqrt{37}}}{\sqrt{37}+4+\sqrt{61}}-1\\\frac{{-\sqrt{61}+24-5\sqrt{37}}}{\sqrt{37}+4+\sqrt{61}}+1}}{\norm{\myvec{-5\\7}} \norm{ \myvec{\frac{{\sqrt{61}-16-3\sqrt{37}}}{\sqrt{37}+4+\sqrt{61}}-1\\\frac{{-\sqrt{61}+24-5\sqrt{37}}}{\sqrt{37}+4+\sqrt{61}}+1}}}
\end{align}
on simplifying $\vec{I}$, we get
\begin{align}
\vec{I} &= \myvec{-1.47756\\-0.79495}\\
&= \frac{\myvec{-5&7}\myvec{-1.47756-1\\-0.79495+1}}{\norm{\myvec{-5\\7}} \norm{ \myvec{-1.47756-1\\-0.79495+1}}}\\
&= \frac{\myvec{-5&7}\myvec{-2.47756\\0.20505}}{\norm{\myvec{-5\\7}} \norm{ \myvec{-2.47756\\0.20505}}}\\
&= \frac{13.82315}{\norm{\myvec{-5\\7}} \norm{\myvec{-2.47756\\0.20505}}}
\end{align}
from (1.1.2.1) length of the side $AB$
\begin{align}
%\intertext{as we know from (1.1.2.1) length of the side $AB$}
%||\vec{A} - \vec{B}|| = \sqrt{(\vec{B}-\vec{A})^T\vec{B}-\vec{A}} &(1.1.2.1)\\
\norm{\vec{A} - \vec{B}} &= \sqrt{(\vec{B}-\vec{A}){^\top}\vec{B}-\vec{A}}\\
&= \frac{13.82315}{\sqrt{74}\sqrt{6.2010538036}}\\
\implies \cos{\angle BAI} &= 0.64529 \\
\implies \angle BAI &= \cos^{-1}{0.64529}\\
&= 49.7311 \label{eq:angle BAI}
\end{align}
consider RHS:
\begin{align}
\cos{\angle CAI} &= \frac{(\vec{I} - \vec{A})^{\top}\vec{C} - \vec{A}}{\norm{\vec{I} - \vec{A}} \norm{ \vec{C} - \vec{A}}}\\
&= \frac{\brak{\myvec{\frac{{\sqrt{61}-16-3\sqrt{37}}}{\sqrt{37}+4+\sqrt{61}}\\\frac{{-\sqrt{61}+24-5\sqrt{37}}}{\sqrt{37}+4+\sqrt{61}}} - \myvec{1\\-1}}^{\top}\myvec{-3\\-5} - \myvec{1\\-1}}{\norm{\myvec{\frac{{\sqrt{61}-16-3\sqrt{37}}}{\sqrt{37}+4+\sqrt{61}}\\ \frac{{-\sqrt{61}+24-5\sqrt{37}}}{\sqrt{37}+4+\sqrt{61}}} - \myvec{1\\-1}} \norm{\myvec{-3\\-5} - \myvec{1\\-1}}}\\
&= \frac{\brak{\myvec{-1.47756\\-0.79495} - \myvec{1\\-1}}^{\top}\myvec{-4\\-4}}{\norm{\myvec{-1.47756\\-0.79495} - \myvec{1\\-1}} \norm{ \myvec{-4\\-4}}}\\
&= \frac{-4\myvec{-2.47756\\0.20505}^{\top}\myvec{1\\1}}{4\norm{\myvec{-2.47756\\0.20505}} \norm{ \myvec{1\\1}}}\\
&= -\frac{\myvec{-2.47756&0.20505}\myvec{1\\1}}{\sqrt{\myvec{-2.47756&0.20505}\myvec{-2.47756\\0.20505}}\sqrt{\myvec{1&1}\myvec{1\\1}}}\\
&= \frac{2.27251}{\sqrt{6.2010538036}\sqrt{2}}\\
\implies \cos{\angle CAI} &= 0.64529 \\
\implies \angle CAI &= \cos^{-1}{0.64529}\\
&= 49.7311 \label{eq:angle CAI}
\end{align}

Therefore from the equations \eqref{eq:angle BAI} and \eqref{eq:angle CAI}, we get:
\begin{align}
\angle BAI &= \angle CAI\\
\therefore LHS &= RHS
\end{align}
Hence we have verified that \begin{align}\angle BAI &= \angle CAI.\end{align}

	\item Find the distance from $\vec{I}$ to $BC$.  
  \\
		\iffalse
\documentclass[journal,12pt,twocolumn]{IEEEtran}
\usepackage{cite}
\usepackage{amsmath,amssymb,amsfonts,amsthm}
\usepackage{algorithmic}
\usepackage{graphicx}
\usepackage{textcomp}
\usepackage{xcolor}
\usepackage{txfonts}
\usepackage{listings}
\usepackage{enumitem}
\usepackage{mathtools}
\usepackage{gensymb}
\usepackage[breaklinks=true]{hyperref}
\usepackage{tkz-euclide} % loads  TikZ and tkz-base
\usepackage{listings}
\usepackage{float}


\begin{document}
\providecommand{\pr}[1]{\ensuremath{\Pr\left(#1\right)}}
\providecommand{\prt}[2]{\ensuremath{p_{#1}^{\left(#2\right)} }}        % own macro for this question
\providecommand{\qfunc}[1]{\ensuremath{Q\left(#1\right)}}
\providecommand{\sbrak}[1]{\ensuremath{{}\left[#1\right]}}
\providecommand{\lsbrak}[1]{\ensuremath{{}\left[#1\right.}}
\providecommand{\rsbrak}[1]{\ensuremath{{}\left.#1\right]}}
\providecommand{\brak}[1]{\ensuremath{\left(#1\right)}}
\providecommand{\lbrak}[1]{\ensuremath{\left(#1\right.}}
\providecommand{\rbrak}[1]{\ensuremath{\left.#1\right)}}
\providecommand{\cbrak}[1]{\ensuremath{\left\{#1\right\}}}
\providecommand{\lcbrak}[1]{\ensuremath{\left\{#1\right.}}
\providecommand{\rcbrak}[1]{\ensuremath{\left.#1\right\}}}
\newcommand{\sgn}{\mathop{\mathrm{sgn}}}
\providecommand{\abs}[1]{\left\vert#1\right\vert}
\providecommand{\res}[1]{\Res\displaylimits_{#1}} 
\providecommand{\norm}[1]{\left\lVert#1\right\rVert}
%\providecommand{\norm}[1]{\lVert#1\rVert}
\providecommand{\mtx}[1]{\mathbf{#1}}
\providecommand{\mean}[1]{E\left[ #1 \right]}
\providecommand{\cond}[2]{#1\middle|#2}
\providecommand{\fourier}{\overset{\mathcal{F}}{ \rightleftharpoons}}
\newenvironment{amatrix}[1]{%
  \left(\begin{array}{@{}*{#1}{c}|c@{}}
}{%
  \end{array}\right)
}
%\providecommand{\hilbert}{\overset{\mathcal{H}}{ \rightleftharpoons}}
%\providecommand{\system}{\overset{\mathcal{H}}{ \longleftrightarrow}}
	%\newcommand{\solution}[2]{\textbf{Solution:}{#1}}
\newcommand{\solution}{\noindent \textbf{Solution: }}
\newcommand{\cosec}{\,\text{cosec}\,}
\providecommand{\dec}[2]{\ensuremath{\overset{#1}{\underset{#2}{\gtrless}}}}
\newcommand{\myvec}[1]{\ensuremath{\begin{pmatrix}#1\end{pmatrix}}}
\newcommand{\mydet}[1]{\ensuremath{\begin{vmatrix}#1\end{vmatrix}}}
\newcommand{\myaugvec}[2]{\ensuremath{\begin{amatrix}{#1}#2\end{amatrix}}}
\providecommand{\rank}{\text{rank}}
\providecommand{\pr}[1]{\ensuremath{\Pr\left(#1\right)}}
\providecommand{\qfunc}[1]{\ensuremath{Q\left(#1\right)}}
	\newcommand*{\permcomb}[4][0mu]{{{}^{#3}\mkern#1#2_{#4}}}
\newcommand*{\perm}[1][-3mu]{\permcomb[#1]{P}}
\newcommand*{\comb}[1][-1mu]{\permcomb[#1]{C}}
\providecommand{\qfunc}[1]{\ensuremath{Q\left(#1\right)}}
\providecommand{\gauss}[2]{\mathcal{N}\ensuremath{\left(#1,#2\right)}}
\providecommand{\diff}[2]{\ensuremath{\frac{d{#1}}{d{#2}}}}
\providecommand{\myceil}[1]{\left \lceil #1 \right \rceil }
\newcommand\figref{Fig.~\ref}
\newcommand\tabref{Table~\ref}
\newcommand{\sinc}{\,\text{sinc}\,}
\newcommand{\rect}{\,\text{rect}\,}
%%
%	%\newcommand{\solution}[2]{\textbf{Solution:}{#1}}
%\newcommand{\solution}{\noindent \textbf{Solution: }}
%\newcommand{\cosec}{\,\text{cosec}\,}
%\numberwithin{equation}{section}
%\numberwithin{equation}{subsection}
%\numberwithin{problem}{section}
%\numberwithin{definition}{section}
%\makeatletter
%\@addtoreset{figure}{problem}
%\makeatother

%\let\StandardTheFigure\thefigure
\let\vec\mathbf

\bibliographystyle{IEEEtran}


\vspace{3cm}


\textbf{1.5.4}
Find distance from $\vec{I}$ to $BC$. \\
\fi
\solution
Given: 
\begin{align}
\vec{A}=\myvec{1\\-1} \\
\vec{B}=\myvec{-4\\6} \\
\vec{C}=\myvec{-3\\-5}
\end{align}
We know incentre \begin{align}
\vec{I} = \frac{1}{\sqrt{37} + 4 + \sqrt{61}} \myvec{\sqrt{61} - 16 - 3\sqrt{37} \\ -\sqrt{61} + 24 - 5\sqrt{37}}
\end{align}
Equation of $BC$:  
\begin{align}
\vec{n} ^\top \vec{x} &= c \\ 
\myvec{11 \\ 1}^\top \vec x &= -38 
\end{align}
Distance from $\vec{I}$ to $BC$
\begin{align}
&=\frac{\abs{\vec{n}^\top \vec{I} - c}}{\norm{\vec{n}}} \\
&=\frac{\abs{\myvec{11\\1}^\top \frac{1}{\sqrt{37} + 4 + \sqrt{61}} \myvec{\sqrt{61} - 16 - 3\sqrt{37} \\ -\sqrt{61} + 24 - 5\sqrt{37}} +38}}{\norm{\myvec{11\\1}}} \\
&=\frac{\abs{\frac{\myvec{11&1}\myvec{\sqrt{61} - 16 - 3\sqrt{37} \\ -\sqrt{61} + 24 - 5\sqrt{37}}}{\sqrt{37} + 4 + \sqrt{61}}+38}}{\sqrt{122}}\\
&=\frac{\abs{\frac{10\sqrt{61}-152-38\sqrt{37}}{\sqrt{37} + 4 + \sqrt{61}}+38}}{\sqrt{122}}\\
&=\frac{48\sqrt{61}}{(\sqrt{37}+4+\sqrt{61})\sqrt{122}}\\
&=1.8968 %verified with python
\end{align}

 
	\item Repeat the above exercise for the sides $AB$ and $AC$.
 \\     \iffalse
\documentclass[]{article}
\usepackage{amsfonts, amsmath, amssymb}
\usepackage[]{setspace}
\doublespacing

\begin{document}
\providecommand{\pr}[1]{\ensuremath{\Pr\left(#1\right)}}
\providecommand{\prt}[2]{\ensuremath{p_{#1}^{\left(#2\right)} }}        % own macro for this question
\providecommand{\qfunc}[1]{\ensuremath{Q\left(#1\right)}}
\providecommand{\sbrak}[1]{\ensuremath{{}\left[#1\right]}}
\providecommand{\lsbrak}[1]{\ensuremath{{}\left[#1\right.}}
\providecommand{\rsbrak}[1]{\ensuremath{{}\left.#1\right]}}
\providecommand{\brak}[1]{\ensuremath{\left(#1\right)}}
\providecommand{\lbrak}[1]{\ensuremath{\left(#1\right.}}
\providecommand{\rbrak}[1]{\ensuremath{\left.#1\right)}}
\providecommand{\cbrak}[1]{\ensuremath{\left\{#1\right\}}}
\providecommand{\lcbrak}[1]{\ensuremath{\left\{#1\right.}}
\providecommand{\rcbrak}[1]{\ensuremath{\left.#1\right\}}}
\newcommand{\sgn}{\mathop{\mathrm{sgn}}}
\providecommand{\abs}[1]{\left\vert#1\right\vert}
\providecommand{\res}[1]{\Res\displaylimits_{#1}} 
\providecommand{\norm}[1]{\left\lVert#1\right\rVert}
%\providecommand{\norm}[1]{\lVert#1\rVert}
\providecommand{\mtx}[1]{\mathbf{#1}}
\providecommand{\mean}[1]{E\left[ #1 \right]}
\providecommand{\cond}[2]{#1\middle|#2}
\providecommand{\fourier}{\overset{\mathcal{F}}{ \rightleftharpoons}}
\newenvironment{amatrix}[1]{%
  \left(\begin{array}{@{}*{#1}{c}|c@{}}
}{%
  \end{array}\right)
}
%\providecommand{\hilbert}{\overset{\mathcal{H}}{ \rightleftharpoons}}
%\providecommand{\system}{\overset{\mathcal{H}}{ \longleftrightarrow}}
	%\newcommand{\solution}[2]{\textbf{Solution:}{#1}}
\newcommand{\solution}{\noindent \textbf{Solution: }}
\newcommand{\cosec}{\,\text{cosec}\,}
\providecommand{\dec}[2]{\ensuremath{\overset{#1}{\underset{#2}{\gtrless}}}}
\newcommand{\myvec}[1]{\ensuremath{\begin{pmatrix}#1\end{pmatrix}}}
\newcommand{\mydet}[1]{\ensuremath{\begin{vmatrix}#1\end{vmatrix}}}
\newcommand{\myaugvec}[2]{\ensuremath{\begin{amatrix}{#1}#2\end{amatrix}}}
\providecommand{\rank}{\text{rank}}
\providecommand{\pr}[1]{\ensuremath{\Pr\left(#1\right)}}
\providecommand{\qfunc}[1]{\ensuremath{Q\left(#1\right)}}
	\newcommand*{\permcomb}[4][0mu]{{{}^{#3}\mkern#1#2_{#4}}}
\newcommand*{\perm}[1][-3mu]{\permcomb[#1]{P}}
\newcommand*{\comb}[1][-1mu]{\permcomb[#1]{C}}
\providecommand{\qfunc}[1]{\ensuremath{Q\left(#1\right)}}
\providecommand{\gauss}[2]{\mathcal{N}\ensuremath{\left(#1,#2\right)}}
\providecommand{\diff}[2]{\ensuremath{\frac{d{#1}}{d{#2}}}}
\providecommand{\myceil}[1]{\left \lceil #1 \right \rceil }
\newcommand\figref{Fig.~\ref}
\newcommand\tabref{Table~\ref}
\newcommand{\sinc}{\,\text{sinc}\,}
\newcommand{\rect}{\,\text{rect}\,}
%%
%	%\newcommand{\solution}[2]{\textbf{Solution:}{#1}}
%\newcommand{\solution}{\noindent \textbf{Solution: }}
%\newcommand{\cosec}{\,\text{cosec}\,}
%\numberwithin{equation}{section}
%\numberwithin{equation}{subsection}
%\numberwithin{problem}{section}
%\numberwithin{definition}{section}
%\makeatletter
%\@addtoreset{figure}{problem}
%\makeatother

%\let\StandardTheFigure\thefigure
\let\vec\mathbf

\fi
1.5.5 Repeat the above excercise for the sides $AB$ and $AC$.

\solution  
We know the value of $\vec{I}$ is
\begin{align}
\vec{I} &= \frac{1}{\sqrt{37} + 4 + \sqrt{61}} \myvec{\sqrt{61} - 16 - 3\sqrt{37}\\ -\sqrt{61} + 24 - 5\sqrt{37}}
\end{align}
from the problem 1.5.2 .
\begin{enumerate}
\item {The equation of $AB$ is:
\begin{align}
\myvec{7&5}\vec{x} - 2=0
\end{align}

Let $r_1$ be the distance between $\vec{I}$ and $AB$, then
\begin{align}
r_1 &= \frac{\abs{\myvec{7&5} \vec{I} - 2}}{\norm{\myvec{7 \\ 5}}} \\
&= \frac{\abs{\frac{1}{\sqrt{37} + 4 + \sqrt{61}} \myvec{7&5} \myvec{\sqrt{61} - 16 - 3\sqrt{37}\\ -\sqrt{61} + 24 - 5\sqrt{37}} - 2}}{\sqrt{{7}^2 + {5}^2}} \\
&= \frac{\frac{2\sqrt{61} - 46\sqrt{37} + 8}{\sqrt{37} + 4 + \sqrt{61}} - 2}{\sqrt{74}} \\
&= \frac{48\sqrt{37}}{\sqrt{74} {(\sqrt{37} + 4 + \sqrt{61}})} \\
&= \frac{48}{\sqrt{2}(\sqrt{37} + 4 + \sqrt{61})}\\
&= \frac{24\sqrt{2}}{\sqrt{37}+4+\sqrt{61}}\\
&= 1.8969                                        
\end{align}}
\item{Similarly, the equation of $AC$ is
\begin{align}
\myvec{4&-4}\vec{x} - 8=0
\end{align}

Let $r_2$ be the distance between $\vec{I}$ and $AC$, then
\begin{align}
r_2 &= \frac{\abs{\myvec{4&-4} \vec{I} - 8}}{\norm{\myvec{4 \\ -4}}} \\
&= \frac{\abs{\frac{1}{\sqrt{37} + 4 + \sqrt{61}} \myvec{4&-4} \myvec{\sqrt{61} - 16 - 3\sqrt{37}\\ -\sqrt{61} + 24 - 5\sqrt{37}} - 8}}{\sqrt{{4}^2 + {(-4)}^2}} \\
&= \frac{\abs{\frac{8\sqrt{61} + 8\sqrt{37} - 160}{\sqrt{37} + 4 + \sqrt{61}} - 8}}{4\sqrt{2}} \\
&= \frac{192}{4\sqrt{2} {(\sqrt{37} + 4 + \sqrt{61}})} \\
&= \frac{48}{\sqrt{2}(\sqrt{37} + 4 + \sqrt{61})}\\
&= \frac{24\sqrt{2}}{\sqrt{37}+4+\sqrt{61}}\\
&= 1.8969       
\end{align} }
\end{enumerate}


	\item This distance is known as the {\em inradius} $r$.
	\item Draw a circle with center $\vec{I}$ and radius $r$.  $\vec{I}$ is known as the {\em incentre}.
	\\	\solution
Let 
\begin{align}
\vec{x} &= \vec{B} + k{\vec{m}}\label{eq:8}
\end{align}
Substituting \eqref{eq:8} in \eqref{eq:incircle}
\begin{align}
  \norm{ \vec{B} + k{\vec{m}}- \vec{I} }^2 &= r^2 \\
\implies   \brak{\vec{B} + k{\vec{m}}- \vec{I}}^{\top} \brak{\vec{B} + k{\vec{m}}- \vec{I}} &= r^2 
\\
	\text{or, }
	k^2\norm{\vec{m}}^2 +2k{\vec{m}^{\top}}\brak{{\vec{B}-\vec{I}}}+\norm{\vec{B}-\vec{I}}^2 &= 0
	\label{eq:incircle-disc}
\end{align}
It can be easily verified that 
\begin{align}
\cbrak{2\vec{m}^{\top}\brak{\vec{B}-\vec{I}}}^2
= 
	\norm{\vec{m}}^2\norm{\vec{B}-\vec{I}}^2
\end{align}
which implies that the discriminant of 
	\eqref{eq:incircle-disc}
	is 0.  Thus, BC intersects the circle at only one point.


	\item The equation of the {\em incircle} is given by 
		\begin{align}
			\norm{\vec{x}-\vec{I}}^2 = r^2
		\end{align}
		Find the parameteric equation of $BC$ and use it to verify that $BC$ intersects the incircle at exactly one point $\vec{D}_3$.  $BC$ is defined to be a {\em tangent} to the incircle.  $\vec{D}_3$ is defined to be {\em point of contact}.
	\\
		\solution
Since
	\eqref{eq:incircle-disc}
	has only one root, 
\begin{align}
k=-\frac{\vec{m}^{\top}\brak{\vec{B}-\vec{I}}}{{\ \norm{\vec{m}}^2}}
\end{align}
Substituing the above in 
\eqref{eq:8},
\begin{align}
	\vec{D_{3}} = \vec{B} -\frac{\vec{m}^{\top}\brak{\vec{B}-\vec{I}}}{{\ \norm{\vec{m}}^2}} \vec{m}
\end{align}


  \item Find the other points of contact $\vec{E}_3$ and $\vec{F}_3$.
  \\
		\iffalse
\let\negmedspace\undefined
\let\negthickspace\undefined
\documentclass[journal,12pt,twocolumn]{IEEEtran}
\usepackage{cite}
\usepackage{amsmath,amssymb,amsfonts,amsthm}
\usepackage{algorithmic}
\usepackage{graphicx}
\usepackage{textcomp}
\usepackage{xcolor}
\usepackage{txfonts}
\usepackage{listings}
\usepackage{enumitem}
\usepackage{mathtools}
\usepackage{gensymb}
\usepackage[breaklinks=true]{hyperref}
\usepackage{tkz-euclide} 
\usepackage{listings}
\usepackage{gvv}
%
%\usepackage{setspace}
%\usepackage{gensymb}
%\doublespacing
%\singlespacing

%\usepackage{graphicx}
%\usepackage{amssymb}
%\usepackage{relsize}
%\usepackage[cmex10]{amsmath}
%\usepackage{amsthm}
%\interdisplaylinepenalty=2500
%\savesymbol{iint}
%\usepackage{txfonts}
%\restoresymbol{TXF}{iint}
%\usepackage{wasysym}
%\usepackage{amsthm}
%\usepackage{iithtlc}
%\usepackage{mathrsfs}
%\usepackage{txfonts}
%\usepackage{stfloats}
%\usepackage{bm}
%\usepackage{cite}
%\usepackage{cases}
%\usepackage{subfig}
%\usepackage{xtab}
%\usepackage{longtable}
%\usepackage{multirow}
%\usepackage{algorithm}
%\usepackage{algpseudocode}
%\usepackage{enumitem}
%\usepackage{mathtools}
%\usepackage{tikz}
%\usepackage{circuitikz}
%\usepackage{verbatim}
%\usepackage{tfrupee}
%\usepackage{stmaryrd}
%\usetkzobj{all}
%    \usepackage{color}                                            %%
%    \usepackage{array}                                            %%
%    \usepackage{longtable}                                        %%
%    \usepackage{calc}                                             %%
%    \usepackage{multirow}                                         %%
%    \usepackage{hhline}                                           %%
%    \usepackage{ifthen}                                           %%
  %optionally (for landscape tables embedded in another document): %%
%    \usepackage{lscape}     
%\usepackage{multicol}
%\usepackage{chngcntr}
%\usepackage{enumerate}

%\usepackage{wasysym}
%\documentclass[conference]{IEEEtran}
%\IEEEoverridecommandlockouts
% The preceding line is only needed to identify funding in the first footnote. If that is unneeded, please comment it out.

\newtheorem{theorem}{Theorem}[section]
\newtheorem{problem}{Problem}
\newtheorem{proposition}{Proposition}[section]
\newtheorem{lemma}{Lemma}[section]
\newtheorem{corollary}[theorem]{Corollary}
\newtheorem{example}{Example}[section]
\newtheorem{definition}[problem]{Definition}

\newcommand{\BEQA}{\begin{eqnarray}}
\newcommand{\EEQA}{\end{eqnarray}}
\newcommand{\define}{\stackrel{\triangle}{=}}
\theoremstyle{remark}
\newtheorem{rem}{Remark}


\begin{document}


\bibliographystyle{IEEEtran}


\vspace{3cm}

\title{
	Question 1.5.9
}

\author{
	EE22BTECH11054 - Umair Parwez
}	

\maketitle
\newpage


\renewcommand{\thefigure}{\theenumi}
\renewcommand{\thetable}{\theenumi}


\textbf{Question 1.5.9:}

Given triangle $ABC$ with vertices, 
\begin{align}
	\vec{A} = \myvec{1\\ -1}, \vec{B} = \myvec{-4\\ 6}, \vec{C} = \myvec{-3\\ -5}
\end{align}
Find the points of contact, $\vec{E}_3$ and $\vec{F}_3$, of the incircle with sides $AC$ and $AB$ respectively.
\fi
\textbf{Solution:}

Required to find points of contact, $\vec{E}_3$ and $\vec{F}_3$, of incircle with sides $AC$ and $AB$ respectively.
From previous questions we know the coordinates of the incircle are : 
\begin{align}
	\vec{I} &= 
	\myvec {
		\frac{-53-11\sqrt{37}+7\sqrt{61}+\sqrt{2257}}{12} \\ 
		\frac{5-\sqrt{37}+5\sqrt{61}-\sqrt{2257}}{12}
	} \\
	&= \myvec{-1.47756 \\ -0.79495}
\end{align}
Radius of incircle is :
    \begin{align}
		r &= \frac{185+41\sqrt{37}-37\sqrt{61}-\sqrt{2257}}{6\sqrt{74}} \\
		&= 1.89689
	\end{align}
Equation of incircle is : 
\begin{align}
	{\norm{ \vec{x}-\vec{I} }}^2 = {r}^2 \label{eq:circle}
\end{align}
points $\vec{A}$, $\vec{B}$ and $\vec{C}$ are : 
\begin{align}
	\vec{A} = \myvec{
		1\\
		-1
	}, 
	\vec{B} = \myvec{
		-4\\
		6
	}, 
	\vec{C} = \myvec{
		-3\\
		-5
	}
\end{align}
Parametric equation of any line is of the form :
\begin{align}
	\vec{x} = \vec{A} + k\vec{m} \label{eq:param}
\end{align}
We can substitute \eqref{eq:param} in \eqref{eq:circle}, and we get : 
\begin{align}
	{\norm{ \vec{A}+k\vec{m}-\vec{I} }}^2 &= {r}^2 \\
	\implies (k\vec{m} + (\vec{A} - \vec{I}))(k\vec{m} + (\vec{A} - \vec{I})) &= {r}^2 \\
	\implies k^2 \norm{\vec{m}}^2 + 2k\vec{m}^{\top} (\vec{A}-\vec{I}) + \norm{\vec{A}-\vec{I}}^2 &= {r}^2 
\end{align}
The discriminant of the above quadratic is,
\begin{align}
	\Delta = 4(\vec{m}^{\top}(\vec{A}-\vec{I}))^2 - 4\norm{\vec{m}}^2(\norm{\vec{A}-\vec{I}}^2 - r^2) \label{eq:Delta}
\end{align} 
If,
\begin{align}
	\Delta = 0
\end{align}
Then the line given by \eqref{eq:param} is tangent to the incircle and the value of $k$ will be given by :
\begin{align}
	k = -\frac{\vec{m}^{\top}(\vec{A} - \vec{I})}{\norm{\vec{m}}^2} \label{eq:k}
\end{align}
Upon substituting $k$ back into \eqref{eq:param}, we will obtain the point of contact of the line with the incircle.
\begin{enumerate}
	\item Finding $\vec{E}_3$ :\\
		Parametric equation of $AC$ is,
		\begin{align}
			\vec{x} = \vec{A} + k\vec{m} \label{eq:AC}
		\end{align}
		where,
		\begin{align}
			\vec{m} = \vec{A} - \vec{C}
		\end{align}	
		First we confirm value of $\Delta$ for $AC$,
		\begin{align}
			\Delta &= 4(\vec{m}^{\top}(\vec{A}-\vec{I}))^2 - 4\norm{\vec{m}}^2(\norm{\vec{A}-\vec{I}}^2 - r^2)\\	
			&= 4(9.09004)^2 - 4(32)(6.18035-3.59819) \\
			&= 0
		\end{align}
		Therefore, $AC$ is tangent to the incircle, and value of $k$ is given by, 
		\begin{align}
			k &= -\frac{\vec{m}^{\top}(\vec{A} - \vec{I})}{\norm{\vec{m}}^2} \\
			&= -\frac{(\vec{A}-\vec{C})^{\top}(\vec{A} - \vec{I})}{\norm{\vec{A}-\vec{C}}^2}
		\end{align}
		Substituting the values, we get,
		\begin{align}
			k = \frac{-4-\sqrt{37}+\sqrt{61}}{2} \label{k:AC} 
		\end{align}
		Substituting \eqref{k:AC} into \eqref{eq:AC}, we get the value of $\vec{E}_3$,
		\begin{align}
			\vec{E}_3 = \myvec{
				\frac{-2-\sqrt{37}+\sqrt{61}}{2}\\ 
				\frac{-6-\sqrt{37}+\sqrt{61}}{2}
			}
		\end{align}
	\item Finding $\vec{F}_3$ :\\
		Parametric equation of $AB$ is,
		\begin{align}
			\vec{x} = \vec{A} + k\vec{m} \label{eq:AB}
		\end{align}
		where,
		\begin{align}
			\vec{m} = \vec{A} - \vec{B}
		\end{align}	
		First we confirm value of $\Delta$ for $AB$,
		\begin{align}
			\Delta &= 4(\vec{m}^{\top}(\vec{A}-\vec{I}))^2 - 4\norm{\vec{m}}^2(\norm{\vec{A}-\vec{I}}^2 - r^2)\\	
			&= 4(13.82315)^2 - 4(74)(6.18035-3.59819) \\
			&= 0
		\end{align}
		Therefore, $AB$ is tangent to the incircle, and value of $k$ is given by, 
		\begin{align}
			k &= -\frac{\vec{m}^{\top}(\vec{A} - \vec{I})}{\norm{\vec{m}}^2} \\
			&= -\frac{(\vec{A}-\vec{B})^{\top}(\vec{A} - \vec{I})}{\norm{\vec{A}-\vec{B}}^2}
		\end{align}
		Substituting the values, we get,
		\begin{align}
			k = \frac{-37-4\sqrt{37}+\sqrt{2257}}{74} \label{k:AB} 
		\end{align}
		Substituting \eqref{k:AB} into \eqref{eq:AB}, we get the value of $\vec{F}_3$,
		\begin{align}
			\vec{F}_3 = \myvec{
				\frac{-111-20\sqrt{37}+5\sqrt{2257}}{74}\\ 
				\frac{185+28\sqrt{37}-7\sqrt{2257}}{74}
			}
		\end{align}
\end{enumerate}
Diagram is shown below,
\begin{figure}[h]
	\centering
	\includegraphics[width=\columnwidth]{./figs/Diagram.png}
	\caption{Points of contact of incircle}
	\label{fig:Incircle}
\end{figure}




	\item Verify that 
		\begin{align}
			AE_3 = AF_3=m, BD_3 = BF_3=n, CD_3 = CE_3=p.
		\end{align}
  \\	\iffalse
\let\negmedspace\undefined
\let\negthickspace\undefined
\documentclass[journal,12pt,twocolumn]{IEEEtran}
\usepackage{cite}
\usepackage{amsmath,amssymb,amsfonts,amsthm}
\usepackage{algorithmic}
\usepackage{graphicx}
\usepackage{textcomp}
\usepackage{xcolor}
\usepackage{txfonts}
\usepackage{listings}
\usepackage{enumitem}
\usepackage{mathtools}
\usepackage{gensymb}
\usepackage[breaklinks=true]{hyperref}
\usepackage{tkz-euclide} % loads  TikZ and tkz-base
\usepackage{listings}
\usepackage{gvv}
%
%\usepackage{setspace}
%\usepackage{gensymb}
%\doublespacing
%\singlespacing

%\usepackage{graphicx}
%\usepackage{amssymb}
%\usepackage{relsize}
%\usepackage[cmex10]{amsmath}
%\usepackage{amsthm}
%\interdisplaylinepenalty=2500
%\savesymbol{iint}
%\usepackage{txfonts}
%\restoresymbol{TXF}{iint}
%\usepackage{wasysym}
%\usepackage{amsthm}
%\usepackage{iithtlc}
%\usepackage{mathrsfs}
%\usepackage{txfonts}
%\usepackage{stfloats}
%\usepackage{bm}
%\usepackage{cite}
%\usepackage{cases}
%\usepackage{subfig}
%\usepackage{xtab}
%\usepackage{longtable}
%\usepackage{multirow}
%\usepackage{algorithm}
%\usepackage{algpseudocode}
%\usepackage{enumitem}
%\usepackage{mathtools}
%\usepackage{tikz}
%\usepackage{circuitikz}
%\usepackage{verbatim}
%\usepackage{tfrupee}
%\usepackage{stmaryrd}
%\usetkzobj{all}
%    \usepackage{color}                                            %%
%    \usepackage{array}                                            %%
%    \usepackage{longtable}                                        %%
%    \usepackage{calc}                                             %%
%    \usepackage{multirow}                                         %%
%    \usepackage{hhline}                                           %%
%    \usepackage{ifthen}                                           %%
  %optionally (for landscape tables embedded in another document): %%
%    \usepackage{lscape}     
%\usepackage{multicol}
%\usepackage{chngcntr}
%\usepackage{enumerate}

%\usepackage{wasysym}
%\documentclass[conference]{IEEEtran}
%\IEEEoverridecommandlockouts
% The preceding line is only needed to identify funding in the first footnote. If that is unneeded, please comment it out.

\newtheorem{theorem}{Theorem}[section]
\newtheorem{problem}{Problem}
\newtheorem{proposition}{Proposition}[section]
\newtheorem{lemma}{Lemma}[section]
\newtheorem{corollary}[theorem]{Corollary}
\newtheorem{example}{Example}[section]
\newtheorem{definition}[problem]{Definition}
%\newtheorem{thm}{Theorem}[section] 
%\newtheorem{defn}[thm]{Definition}
%\newtheorem{algorithm}{Algorithm}[section]
%\newtheorem{cor}{Corollary}
\newcommand{\BEQA}{\begin{eqnarray}}
\newcommand{\EEQA}{\end{eqnarray}}
\newcommand{\define}{\stackrel{\triangle}{=}}
\theoremstyle{remark}
\newtheorem{rem}{Remark}

%\bibliographystyle{ieeetr}
\begin{document}
%

\bibliographystyle{IEEEtran}


\vspace{3cm}

\title{
%	\logo{
	Solution to 1.5.10
%	}
}


	
}	
%\title{
%	\logo{Matrix Analysis through Octave}{\begin{center}\includegraphics[scale=.24]{tlc}\end{center}}{}{HAMDSP}
%}


% paper title
% can use linebreaks \\ within to get better formatting as desired
%\title{Matrix Analysis through Octave}
%
%
% author names and IEEE memberships
% note positions of commas and nonbreaking spaces ( ~ ) LaTeX will not break
% a structure at a ~ so this keeps an author's name from being broken across
% two lines.
% use \thanks{} to gain access to the first footnote area
% a separate \thanks must be used for each paragraph as LaTeX2e's \thanks
% was not built to handle multiple paragraphs
%

%\author{<-this % stops a space
%\thanks{}}
%}
% note the % following the last \IEEEmembership and also \thanks - 
% these prevent an unwanted space from occurring between the last author name
% and the end of the author line. i.e., if you had this:
% 
% \author{....lastname \thanks{...} \thanks{...} }
%                     ^------------^------------^----Do not want these spaces!
%
% a space would be appended to the last name and could cause every name on that
% line to be shifted left slightly. This is one of those "LaTeX things". For
% instance, "\textbf{A} \textbf{B}" will typeset as "A B" not "AB". To get
% "AB" then you have to do: "\textbf{A}\textbf{B}"
% \thanks is no different in this regard, so shield the last } of each \thanks
% that ends a line with a % and do not let a space in before the next \thanks.
% Spaces after \IEEEmembership other than the last one are OK (and needed) as
% you are supposed to have spaces between the names. For what it is worth,
% this is a minor point as most people would not even notice if the said evil
% space somehow managed to creep in.



% The paper headers
%\markboth{Journal of \LaTeX\ Class Files,~Vol.~6, No.~1, January~2007}%
%{Shell \MakeLowercase{\textit{et al.}}: Bare Demo of IEEEtran.cls for Journals}
% The only time the second header will appear is for the odd numbered pages
% after the title page when using the twoside option.
% 
% *** Note that you probably will NOT want to include the author's ***
% *** name in the headers of peer review papers.                   ***
% You can use \ifCLASSOPTIONpeerreview for conditional compilation here if
% you desire.




% If you want to put a publisher's ID mark on the page you can do it like
% this:
%\IEEEpubid{0000--0000/00\$00.00~\copyright~2007 IEEE}
% Remember, if you use this you must call \IEEEpubidadjcol in the second
% column for its text to clear the IEEEpubid mark.



% make the title area
\maketitle

\newpage

%\tableofcontents

\bigskip

\renewcommand{\thefigure}{\theenumi}
\renewcommand{\thetable}{\theenumi}
%\renewcommand{\theequation}{\theenumi}

%\begin{abstract}
%%\boldmath
%In this letter, an algorithm for evaluating the exact analytical bit error rate  (BER)  for the piecewise linear (PL) combiner for  multiple relays is presented. Previous results were available only for upto three relays. The algorithm is unique in the sense that  the actual mathematical expressions, that are prohibitively large, need not be explicitly obtained. The diversity gain due to multiple relays is shown through plots of the analytical BER, well supported by simulations. 
%
%\end{abstract}
% IEEEtran.cls defaults to using nonbold math in the Abstract.
% This preserves the distinction between vectors and scalars. However,
% if the journal you are submitting to favors bold math in the abstract,
% then you can use LaTeX's standard command \boldmath at the very start
% of the abstract to achieve this. Many IEEE journals frown on math
% in the abstract anyway.

% Note that keywords are not normally used for peerreview papers.
%\begin{IEEEkeywords}
%Cooperative diversity, decode and forward, piecewise linear
%\end{IEEEkeywords}

	

% For peer review papers, you can put extra information on the cover
% page as needed:
% \ifCLASSOPTIONpeerreview
% \begin{center} \bfseries EDICS Category: 3-BBND \end{center}
% \fi
%
% For peerreview papers, this IEEEtran command inserts a page break and
% creates the second title. It will be ignored for other modes.
%\IEEEpeerreviewmaketitle
\textbf{Question:}
Verify that:
\begin{align}
	AE_3 = AF_3=m, BD_3 = BF_3=n, CD_3 = CE_3=p.
\end{align}
\fi
\indent\solution
\begin{flushleft}
The coordinates of the points of contact of the circle and the triangle are: 
\end{flushleft}
\begin{align}
	\vec{D}_3 &= \myvec{
		\frac {-366\sqrt{74}-406\sqrt{122}-488\sqrt{32}}{122(\sqrt{74}+\sqrt{32}+\sqrt{122})}\\
		\frac {-610\sqrt{74}-170\sqrt{122}+732\sqrt{32}}{122(\sqrt{74}+\sqrt{32}+\sqrt{122})}
                }
		{{\, \text{from} \;1.5.8}}\\
	 \vec{E}_3 &= \myvec{
                 \frac{-111-20\sqrt{37}+5\sqrt{2257}}{74} \\
         \frac{185+28\sqrt{37}-7\sqrt{2257}}{74}
    }
		{{\, \text{from} \;1.5.9}}\\
	\vec{F}_3 &= \myvec{
      \frac{-2-\sqrt{37}+\sqrt{61}}{2}\\
      \frac{-6-\sqrt{37}+\sqrt{61}}{2}
    }
	{{\, \text{from} \; 1.5.9}}
  \end{align}
Length of line segment between two points is given by:
\begin{enumerate}
	\item \begin{align}
	{AE_3}&=\sqrt{{(\vec{E_3}-\vec A)}^{\top}{(\vec{E_3}-\vec A)}}\\
	\vec{E_3}-\vec A&=\myvec{-0.136-1 \\ 
       -2.136+1}\\
\implies	{AE_3}&=\sqrt{\myvec{-1.136 &
       -1.136}
       \myvec{ -1.136\\ 
       -1.136}} \\
	&= 1.607 
\end{align}
\begin{align}
 	{AF_3}&=\sqrt{{(\vec{F_3}-\vec A)}^{\top}{(\vec{F_3}-\vec A)}}\\ 
	\vec{F_3}-\vec A&=\myvec{0.066-1 \\ 
       0.308+1}\\
\implies	{AF_3}&=\sqrt{\myvec{-0.934 &
       1.308}
	\myvec{-0.934 \\ 
       1.308}} \\
	&=1.607
\end{align}
$\therefore{AE_3}= {AF_3}=m$ is verified.
\item \begin{align}
	{BD_3}&=\sqrt{{(\vec{D_3}-\vec B)}^{\top}{(\vec{D_3}-\vec B)}}\\
 	\vec{D_3}-\vec B&=\myvec{ -3.367 +4\\ 
                -0.967 -6}\\
\implies	{BD_3}&=\sqrt{\myvec{ 0.633  &
       6.967}
       \myvec{ 0.633\\ 
       6.967}} \\
	&=6.995 
\end{align}
\begin{align}
	{BF_3}&=\sqrt{{(\vec{F_3}-\vec B)}^{\top}{(\vec{F_3}-\vec B)}}\\
	\vec{F_3}-\vec B&=\myvec{ 0.066+4 \\
         0.308 -6}\\                              
\implies	{BF_3}&=\sqrt{\myvec{ 4.066 &
       -5.692}
       \myvec{4.066 \\ 
       -5.692}} \\
	&=6.995 
\end{align}
$\therefore {BD_3}= {BF_3}=n$ is verified.
\item \begin{align}
	{CD_3}&=\sqrt{{(\vec{D_3}-\vec C)}^{\top}{(\vec{D_3}-\vec C)}}\\
	\vec{D_3}-\vec C&=\myvec{ -3.367 + 3\\
       -0.967 + 5}\\
\implies	{CD_3}&=\sqrt{\myvec{ -0.367&
       4.033}
       \myvec{ -0.367\\
       4.033}} \\
	&=4.0499 
\end{align}           
\begin{align}
	{CE_3}&=\sqrt{{(\vec{E_3}-\vec C)}^{\top}{(\vec{E_3}-\vec C)}}\\
	\vec{E_3}-\vec C&=\myvec{ -0.136 + 3\\ 
       -2.136+5}\\
\implies	{CE_3}&=\sqrt{\myvec{ 2.864 &
       2.864}
       \myvec{ 2.864\\ 
       2.864}}\\
	&= 4.0499            
\end{align}
$\therefore {CD_3}= {CE_3}=p$ is verified.
\end{enumerate}

	\item Obtain $m,n,p$ in terms of $a,b,c$, the sides of the triangle using a matrix equation.  Obtain the numerical values.
 \\
 		\solution 
From the given information, 
\begin{align}
% 
    a &= m+n,\\
    b &= n+p, \\
    c &= m+p 
\end{align}
which can be expressed as
\begin{align}
\myvec{1&1&0\\0&1&1\\1&0&1\\}\myvec{m\\n\\p} &= \myvec{a\\b\\c}
\\
\implies 
	\myvec{m\\n\\p} &= \myvec{1&1&0\\0&1&1\\1&0&1\\}^{-1}\myvec{a\\b\\c}
\end{align}
Using row reduction,
		\begin{align}
			\augvec{3}{3}{1&1&0 & 1 & 0 & 0\\0&1&1 & 0 & 1 & 0\\1&0&1 & 0 & 0 & 1}
			\xleftrightarrow[]{R_3 \leftarrow R_3 - R_1}
			\augvec{3}{3}{1&1&0 & 1 & 0 & 0\\0&1&1 & 0 & 1 & 0\\0&-1&1 & -1 & 0 & 1}
		\end{align}
		\begin{align}
			\xleftrightarrow[R_1 \leftarrow R_1 - R_2]{R_3 \leftarrow R_3 + R_2}
			\augvec{3}{3}{1&0&-1 & 1 & -1 & 0\\0&1&1 & 0 & 1 & 0\\0&0&2 & -1 & 1 & 1}
		\end{align}
		\begin{align}
			\xleftrightarrow[R_1 \leftarrow 2R_1 + R_3]{R_2 \leftarrow 2R_2 - R_3}
			\augvec{3}{3}{2&0&0 & 1 & -1 & 1\\0&2&0 & 1 & 1 & -1\\0&0&2 & -1 & 1 & 1}
		\end{align}
yielding
		\begin{align}
			\myvec{1&1&0\\0&1&1\\1&0&1\\}^{-1} = 
			\frac{1}{2}\myvec{1 & -1 & 1\\ 1 & 1 & -1\\ -1 & 1 & 1}
		\end{align}
	Therefore,
\begin{align}
\begin{split}
    p&=\frac{c+b-a}{2}
    =\frac{\sqrt{74}+\sqrt{32}-\sqrt{122}}{2}
    \\
    m&=\frac{a+c-b}{2}
    =\frac{\sqrt{74}+\sqrt{122}-\sqrt{32}}{2}
    \\
    n&=\frac{a+b-c}{2}
    =\frac{\sqrt{122}+\sqrt{32}-\sqrt{74}}{2}
\end{split}
	\label{eq:incircle-mnp}
\end{align}
upon substituting from 
		\eqref{eq:geo-norm-ab},
		\eqref{eq:geo-norm-bc}
		and
		\eqref{eq:geo-norm-ca}.

\end{enumerate}

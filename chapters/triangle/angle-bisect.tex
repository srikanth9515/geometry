%\renewcommand{\theequation}{\theenumi}
%\begin{enumerate}[label=\arabic*.,ref=\theenumi]
\begin{enumerate}[label=\thesection.\arabic*.,ref=\thesection.\theenumi]
\numberwithin{equation}{enumi}
	\item Let $\vec{D}_3, \vec{E}_3, \vec{F}_3$, be points on $AB, BC$ and $CA$ respectively such that
		\begin{align}
			BD_3 = BF_3=m, CD_3 = CE_3=n, AE_3 = AF_3=p.
		\end{align}
	Obtain $m,n,p$ in terms of $a,b,c$ obtained in  
		\probref{prob:side-length}.
 \\
 		\solution 
From the given information, 
\begin{align}
% 
    a &= m+n,\\
    b &= n+p, \\
    c &= m+p 
\end{align}
which can be expressed as
\begin{align}
\myvec{1&1&0\\0&1&1\\1&0&1\\}\myvec{m\\n\\p} &= \myvec{a\\b\\c}
\\
\implies 
	\myvec{m\\n\\p} &= \myvec{1&1&0\\0&1&1\\1&0&1\\}^{-1}\myvec{a\\b\\c}
\end{align}
Using row reduction,
		\begin{align}
			\augvec{3}{3}{1&1&0 & 1 & 0 & 0\\0&1&1 & 0 & 1 & 0\\1&0&1 & 0 & 0 & 1}
			\xleftrightarrow[]{R_3 \leftarrow R_3 - R_1}
			\augvec{3}{3}{1&1&0 & 1 & 0 & 0\\0&1&1 & 0 & 1 & 0\\0&-1&1 & -1 & 0 & 1}
		\end{align}
		\begin{align}
			\xleftrightarrow[R_1 \leftarrow R_1 - R_2]{R_3 \leftarrow R_3 + R_2}
			\augvec{3}{3}{1&0&-1 & 1 & -1 & 0\\0&1&1 & 0 & 1 & 0\\0&0&2 & -1 & 1 & 1}
		\end{align}
		\begin{align}
			\xleftrightarrow[R_1 \leftarrow 2R_1 + R_3]{R_2 \leftarrow 2R_2 - R_3}
			\augvec{3}{3}{2&0&0 & 1 & -1 & 1\\0&2&0 & 1 & 1 & -1\\0&0&2 & -1 & 1 & 1}
		\end{align}
yielding
		\begin{align}
			\myvec{1&1&0\\0&1&1\\1&0&1\\}^{-1} = 
			\frac{1}{2}\myvec{1 & -1 & 1\\ 1 & 1 & -1\\ -1 & 1 & 1}
		\end{align}
	Therefore,
\begin{align}
\begin{split}
    p&=\frac{c+b-a}{2}
    =\frac{\sqrt{74}+\sqrt{32}-\sqrt{122}}{2}
    \\
    m&=\frac{a+c-b}{2}
    =\frac{\sqrt{74}+\sqrt{122}-\sqrt{32}}{2}
    \\
    n&=\frac{a+b-c}{2}
    =\frac{\sqrt{122}+\sqrt{32}-\sqrt{74}}{2}
\end{split}
	\label{eq:incircle-mnp}
\end{align}
upon substituting from 
		\eqref{eq:geo-norm-ab},
		\eqref{eq:geo-norm-bc}
		and
		\eqref{eq:geo-norm-ca}.

	\item Using section formula, find 
		\begin{align}
			\vec{D}_3 = \frac{m\vec{C}+n\vec{B}}{m+n},\,
			\vec{E}_3 = \frac{n\vec{A}+p\vec{C}}{n+p},\,
			\vec{F}_3 = \frac{p\vec{B}+m\vec{A}}{p+m}
		\end{align}
	\item Find the circumcentre and circumradius of $\triangle D_3E_3F_3$.  These are the {\em incentre} and {\em inradius} of $\triangle ABC$.
	\item Draw the circumcircle of $\triangle D_3E_3F_3$.  This is known as the {\em incircle} of $\triangle ABC$.
		\\
 		\iffalse
\documentclass[journal,12pt,twocolumn]{IEEEtran}
\usepackage{cite}
\usepackage{amsmath,amssymb,amsfonts,amsthm}
\usepackage{algorithmic}
\usepackage{graphicx}
\usepackage{textcomp}
\usepackage{xcolor}
\usepackage{txfonts}
\usepackage{listings}
\usepackage{enumitem}
\usepackage{mathtools}
\usepackage{gensymb}
\usepackage[breaklinks=true]{hyperref}
\usepackage{tkz-euclide} % loads  TikZ and tkz-base
\usepackage{listings}
\usepackage{float}


\begin{document}
\providecommand{\pr}[1]{\ensuremath{\Pr\left(#1\right)}}
\providecommand{\prt}[2]{\ensuremath{p_{#1}^{\left(#2\right)} }}        % own macro for this question
\providecommand{\qfunc}[1]{\ensuremath{Q\left(#1\right)}}
\providecommand{\sbrak}[1]{\ensuremath{{}\left[#1\right]}}
\providecommand{\lsbrak}[1]{\ensuremath{{}\left[#1\right.}}
\providecommand{\rsbrak}[1]{\ensuremath{{}\left.#1\right]}}
\providecommand{\brak}[1]{\ensuremath{\left(#1\right)}}
\providecommand{\lbrak}[1]{\ensuremath{\left(#1\right.}}
\providecommand{\rbrak}[1]{\ensuremath{\left.#1\right)}}
\providecommand{\cbrak}[1]{\ensuremath{\left\{#1\right\}}}
\providecommand{\lcbrak}[1]{\ensuremath{\left\{#1\right.}}
\providecommand{\rcbrak}[1]{\ensuremath{\left.#1\right\}}}
\newcommand{\sgn}{\mathop{\mathrm{sgn}}}
\providecommand{\abs}[1]{\left\vert#1\right\vert}
\providecommand{\res}[1]{\Res\displaylimits_{#1}} 
\providecommand{\norm}[1]{\left\lVert#1\right\rVert}
%\providecommand{\norm}[1]{\lVert#1\rVert}
\providecommand{\mtx}[1]{\mathbf{#1}}
\providecommand{\mean}[1]{E\left[ #1 \right]}
\providecommand{\cond}[2]{#1\middle|#2}
\providecommand{\fourier}{\overset{\mathcal{F}}{ \rightleftharpoons}}
\newenvironment{amatrix}[1]{%
  \left(\begin{array}{@{}*{#1}{c}|c@{}}
}{%
  \end{array}\right)
}
%\providecommand{\hilbert}{\overset{\mathcal{H}}{ \rightleftharpoons}}
%\providecommand{\system}{\overset{\mathcal{H}}{ \longleftrightarrow}}
	%\newcommand{\solution}[2]{\textbf{Solution:}{#1}}
\newcommand{\solution}{\noindent \textbf{Solution: }}
\newcommand{\cosec}{\,\text{cosec}\,}
\providecommand{\dec}[2]{\ensuremath{\overset{#1}{\underset{#2}{\gtrless}}}}
\newcommand{\myvec}[1]{\ensuremath{\begin{pmatrix}#1\end{pmatrix}}}
\newcommand{\mydet}[1]{\ensuremath{\begin{vmatrix}#1\end{vmatrix}}}
\newcommand{\myaugvec}[2]{\ensuremath{\begin{amatrix}{#1}#2\end{amatrix}}}
\providecommand{\rank}{\text{rank}}
\providecommand{\pr}[1]{\ensuremath{\Pr\left(#1\right)}}
\providecommand{\qfunc}[1]{\ensuremath{Q\left(#1\right)}}
	\newcommand*{\permcomb}[4][0mu]{{{}^{#3}\mkern#1#2_{#4}}}
\newcommand*{\perm}[1][-3mu]{\permcomb[#1]{P}}
\newcommand*{\comb}[1][-1mu]{\permcomb[#1]{C}}
\providecommand{\qfunc}[1]{\ensuremath{Q\left(#1\right)}}
\providecommand{\gauss}[2]{\mathcal{N}\ensuremath{\left(#1,#2\right)}}
\providecommand{\diff}[2]{\ensuremath{\frac{d{#1}}{d{#2}}}}
\providecommand{\myceil}[1]{\left \lceil #1 \right \rceil }
\newcommand\figref{Fig.~\ref}
\newcommand\tabref{Table~\ref}
\newcommand{\sinc}{\,\text{sinc}\,}
\newcommand{\rect}{\,\text{rect}\,}
%%
%	%\newcommand{\solution}[2]{\textbf{Solution:}{#1}}
%\newcommand{\solution}{\noindent \textbf{Solution: }}
%\newcommand{\cosec}{\,\text{cosec}\,}
%\numberwithin{equation}{section}
%\numberwithin{equation}{subsection}
%\numberwithin{problem}{section}
%\numberwithin{definition}{section}
%\makeatletter
%\@addtoreset{figure}{problem}
%\makeatother

%\let\StandardTheFigure\thefigure
\let\vec\mathbf

\bibliographystyle{IEEEtran}


\vspace{3cm}


\textbf{1.5.4}
Find distance from $\vec{I}$ to $BC$. \\
\fi
\solution
Given: 
\begin{align}
\vec{A}=\myvec{1\\-1} \\
\vec{B}=\myvec{-4\\6} \\
\vec{C}=\myvec{-3\\-5}
\end{align}
We know incentre \begin{align}
\vec{I} = \frac{1}{\sqrt{37} + 4 + \sqrt{61}} \myvec{\sqrt{61} - 16 - 3\sqrt{37} \\ -\sqrt{61} + 24 - 5\sqrt{37}}
\end{align}
Equation of $BC$:  
\begin{align}
\vec{n} ^\top \vec{x} &= c \\ 
\myvec{11 \\ 1}^\top \vec x &= -38 
\end{align}
Distance from $\vec{I}$ to $BC$
\begin{align}
&=\frac{\abs{\vec{n}^\top \vec{I} - c}}{\norm{\vec{n}}} \\
&=\frac{\abs{\myvec{11\\1}^\top \frac{1}{\sqrt{37} + 4 + \sqrt{61}} \myvec{\sqrt{61} - 16 - 3\sqrt{37} \\ -\sqrt{61} + 24 - 5\sqrt{37}} +38}}{\norm{\myvec{11\\1}}} \\
&=\frac{\abs{\frac{\myvec{11&1}\myvec{\sqrt{61} - 16 - 3\sqrt{37} \\ -\sqrt{61} + 24 - 5\sqrt{37}}}{\sqrt{37} + 4 + \sqrt{61}}+38}}{\sqrt{122}}\\
&=\frac{\abs{\frac{10\sqrt{61}-152-38\sqrt{37}}{\sqrt{37} + 4 + \sqrt{61}}+38}}{\sqrt{122}}\\
&=\frac{48\sqrt{61}}{(\sqrt{37}+4+\sqrt{61})\sqrt{122}}\\
&=1.8968 %verified with python
\end{align}


	\item Using 
    \eqref{eq:angle2d}
verify that 
		\begin{align}
			\angle BAI = \angle CAI.
		\end{align}
		$AI$ is the bisector of $\angle A$.  
	\item Verify that $BI, CI$ are also the angle bisectors of $\triangle ABC$.

		\iffalse

\item Suppose the equations  given by 
		\begin{align}
			\label{eq:tri-sides}
			\vec{n}_i^{\top}\vec{x}=c_i \quad i = 1, 2, 3 
		\end{align}
		The equations of the respective angle bisectors are then given by 
		\begin{align}
			\frac{\vec{n}_i^{\top}\vec{x}-c_i}{\norm{\vec{n}_i}}
		=
	\pm	\frac{\vec{n}_j^{\top}\vec{x}-c_j}{\norm{\vec{n}_j}}
\quad i \ne j
		\end{align}
		Substitute numerical values and find the equations of the angle bisectors of $A, B$ and $C$.
	\\
		%\textbf{Solution :}
	The parametric equations of sides;
	\begin{align}
	BC:\quad &\myvec{11&1}\vec{x}=-38,\\
	CA:\quad &\myvec{1&-1}\vec{x}=2,\\
	AB:\quad &\myvec{7&5}\vec{x}=2\\	  
	\end{align}
	Using the formula mentioned in the question to find out the angular bisector for sides \text{AB} and \text{AC}, naming the angular bisector $L$ we get,
	\begin{align}
		\frac{\vec{n}_{3}^{\top} \vec{x}-c_{3}}{\norm{\vec{n}_{3}}}=\pm \frac{\vec{n}_{2}^{\top} \vec{x}-c_{2}}{\norm{\vec{n}_{2}}}
	\end{align}
	\begin{figure}
	\centering
	\includegraphics[width=\columnwidth]{solutions/1/5/1/figs/angular_bisector.png}
	\caption{Triangle generated using python}
	\label{fig:angular_bisector}
	\end{figure}
	As we can see we will get 2 solutions for $L$. This is because one of them is internal angular bisector and the other is the external angular bisector. Internal angular bisector can be evaluated if we take + in the above formula.
	Hence, $L$ is given by,
	\begin{align}
		\frac{\vec{n}_{3}^{\top} \vec{x}-c_{3}}{\norm{\vec{n}_{3}}}&=\frac{\vec{n}_{2}^{\top} \vec{x}-c_{2}}{\norm{\vec{n}_{2}}}\\
		\implies \brak{\frac{\vec{n_{3}}}{\norm{\vec{n_{3}}}}-\frac{\vec{n_{3}}}{\norm{\vec{n_{3}}}}} \vec{x}&=\brak{\frac{c_{3}}{\norm{\vec{n_{3}}}}-\frac{c_{2}}{\norm{\vec{n_{2}}}}}\\
		\implies \brak{\frac{\myvec{7&5}}{\sqrt{74}}-\frac{\myvec{1&-1}}{\sqrt{2}}} \vec{x}&=\frac{2}{\sqrt{74}}-\frac{2}{\sqrt{2}}\\
		\implies \myvec{\frac{7-\sqrt{37}}{\sqrt{74}}&\frac{5+\sqrt{37}}{\sqrt{74}}}\vec{x}&=\frac{2-2\sqrt{37}}{\sqrt{74}}
	\end{align}
	Hence, the internal angluar bisector of angle $A$, $L$ will be,
	\begin{align}
		\implies\myvec{\frac{7-\sqrt{37}}{\sqrt{74}}&\frac{5+\sqrt{37}}{\sqrt{74}}} \vec{x}=\frac{2-2\sqrt{37}}{\sqrt{74}}
		\label{eq:1.5.1}
	\end{align}
	

  \iffalse
\documentclass[journal,12pt,twocolumn]{IEEEtran}
\usepackage{cite}
\usepackage{amsmath,amssymb,amsfonts,amsthm}
\usepackage{algorithmic}
\usepackage{graphicx}
\usepackage{textcomp}
\usepackage{xcolor}
\usepackage{txfonts}
\usepackage{listings}
\usepackage{enumitem}
\usepackage{mathtools}
\usepackage{gensymb}
\usepackage[breaklinks=true]{hyperref}
\usepackage{tkz-euclide} % loads  TikZ and tkz-base
\usepackage{listings}
\usepackage{float}


\begin{document}
\providecommand{\pr}[1]{\ensuremath{\Pr\left(#1\right)}}
\providecommand{\prt}[2]{\ensuremath{p_{#1}^{\left(#2\right)} }}        % own macro for this question
\providecommand{\qfunc}[1]{\ensuremath{Q\left(#1\right)}}
\providecommand{\sbrak}[1]{\ensuremath{{}\left[#1\right]}}
\providecommand{\lsbrak}[1]{\ensuremath{{}\left[#1\right.}}
\providecommand{\rsbrak}[1]{\ensuremath{{}\left.#1\right]}}
\providecommand{\brak}[1]{\ensuremath{\left(#1\right)}}
\providecommand{\lbrak}[1]{\ensuremath{\left(#1\right.}}
\providecommand{\rbrak}[1]{\ensuremath{\left.#1\right)}}
\providecommand{\cbrak}[1]{\ensuremath{\left\{#1\right\}}}
\providecommand{\lcbrak}[1]{\ensuremath{\left\{#1\right.}}
\providecommand{\rcbrak}[1]{\ensuremath{\left.#1\right\}}}
\newcommand{\sgn}{\mathop{\mathrm{sgn}}}
\providecommand{\abs}[1]{\left\vert#1\right\vert}
\providecommand{\res}[1]{\Res\displaylimits_{#1}} 
\providecommand{\norm}[1]{\left\lVert#1\right\rVert}
%\providecommand{\norm}[1]{\lVert#1\rVert}
\providecommand{\mtx}[1]{\mathbf{#1}}
\providecommand{\mean}[1]{E\left[ #1 \right]}
\providecommand{\cond}[2]{#1\middle|#2}
\providecommand{\fourier}{\overset{\mathcal{F}}{ \rightleftharpoons}}
\newenvironment{amatrix}[1]{%
  \left(\begin{array}{@{}*{#1}{c}|c@{}}
}{%
  \end{array}\right)
}
%\providecommand{\hilbert}{\overset{\mathcal{H}}{ \rightleftharpoons}}
%\providecommand{\system}{\overset{\mathcal{H}}{ \longleftrightarrow}}
	%\newcommand{\solution}[2]{\textbf{Solution:}{#1}}
\newcommand{\solution}{\noindent \textbf{Solution: }}
\newcommand{\cosec}{\,\text{cosec}\,}
\providecommand{\dec}[2]{\ensuremath{\overset{#1}{\underset{#2}{\gtrless}}}}
\newcommand{\myvec}[1]{\ensuremath{\begin{pmatrix}#1\end{pmatrix}}}
\newcommand{\mydet}[1]{\ensuremath{\begin{vmatrix}#1\end{vmatrix}}}
\newcommand{\myaugvec}[2]{\ensuremath{\begin{amatrix}{#1}#2\end{amatrix}}}
\providecommand{\rank}{\text{rank}}
\providecommand{\pr}[1]{\ensuremath{\Pr\left(#1\right)}}
\providecommand{\qfunc}[1]{\ensuremath{Q\left(#1\right)}}
	\newcommand*{\permcomb}[4][0mu]{{{}^{#3}\mkern#1#2_{#4}}}
\newcommand*{\perm}[1][-3mu]{\permcomb[#1]{P}}
\newcommand*{\comb}[1][-1mu]{\permcomb[#1]{C}}
\providecommand{\qfunc}[1]{\ensuremath{Q\left(#1\right)}}
\providecommand{\gauss}[2]{\mathcal{N}\ensuremath{\left(#1,#2\right)}}
\providecommand{\diff}[2]{\ensuremath{\frac{d{#1}}{d{#2}}}}
\providecommand{\myceil}[1]{\left \lceil #1 \right \rceil }
\newcommand\figref{Fig.~\ref}
\newcommand\tabref{Table~\ref}
\newcommand{\sinc}{\,\text{sinc}\,}
\newcommand{\rect}{\,\text{rect}\,}
%%
%	%\newcommand{\solution}[2]{\textbf{Solution:}{#1}}
%\newcommand{\solution}{\noindent \textbf{Solution: }}
%\newcommand{\cosec}{\,\text{cosec}\,}
%\numberwithin{equation}{section}
%\numberwithin{equation}{subsection}
%\numberwithin{problem}{section}
%\numberwithin{definition}{section}
%\makeatletter
%\@addtoreset{figure}{problem}
%\makeatother

%\let\StandardTheFigure\thefigure
\let\vec\mathbf

\bibliographystyle{IEEEtran}


\vspace{3cm}

\textbf{Question 1.5.1}\\
Suppose the equations $AB, BC$ and $CA$ are respectively given by 
		\begin{align}
			\label{eq:tri-sides}
			\vec{n}_i^{\top}\vec{x}=c_i \quad i = 1, 2, 3 
		\end{align}
		The equations of the respective angle bisectors are then given by 
		\begin{align}
			\frac{\vec{n}_i^{\top}\vec{x}-c_i}{\norm{\vec{n}_i}}
		=
	\pm	\frac{\vec{n}_j^{\top}\vec{x}-c_j}{\norm{\vec{n}_j}}
\quad i \ne j
		\end{align}
		Substitute numerical values and find the equations of the angle bisectors of $A, B$ and $C$.\\
\fi
\solution 
The internal angle bisector is obtained from the set of two bisectors by using:
		\begin{align}
			\frac{\vec{n}_i^{\top}\vec{x}-c_i}{\norm{\vec{n}_i}}
		=
		\frac{\vec{n}_j^{\top}\vec{x}-c_j}{\norm{\vec{n}_j}}
\quad i \ne j	
\end{align}
This can be transformed to the normal equation of angle bisectors as follows 
\begin{align}
       \myvec{\frac{\vec{n}_i^{\top}}{\norm{\vec{n}_i}} - \frac{\vec{n}_j^{\top}}{\norm{\vec{n}_j}}}\vec{x}
       =
       \frac{c_i}{\norm{\vec{n}_i}}-\frac{c_j}{\norm{\vec{n}_j}}
\end{align}
$i$ and $j$ values correspond to the sides including the angle\\
\begin{enumerate}
\item Angle Bisector of $A$
\begin{align}
       \myvec{\frac{\vec{n}_3^{\top}}{\norm{\vec{n}_3}} - \frac{\vec{n}_1^{\top}}{\norm{\vec{n}_1}}}\vec{x}
       =
       \frac{c_3}{\norm{\vec{n}_3}}-\frac{c_1}{\norm{\vec{n}_1}}
\end{align}
on substitution we obtain 
\begin{align}
\myvec{
\frac{7}{\sqrt{74}}-\frac{1}{\sqrt{2}} & \frac{5}{\sqrt{74}}+\frac{1}{\sqrt{2}}\\
}
\vec{x}
=\frac{2}{\sqrt{74}}-\frac{2}{\sqrt{2}}
\end{align}
\item Angle Bisector of $B$
\begin{align}
       \myvec{\frac{\vec{n}_2^{\top}}{\norm{\vec{n}_2}} - \frac{\vec{n}_1^{\top}}{\norm{\vec{n}_1}}}\vec{x}
       =
       \frac{c_2}{\norm{\vec{n}_2}}-\frac{c_1}{\norm{\vec{n}_1}}
\end{align}
on substitution we obtain 
\begin{align}
\myvec{
\frac{11}{\sqrt{122}}+\frac{7}{\sqrt{74}} & \frac{1}{\sqrt{122}}+\frac{5}{\sqrt{74}}\\
}
\vec{x}
=\frac{2}{\sqrt{74}}-\frac{38}{\sqrt{122}}
\end{align}
\item Angle Bisector of $C$
\begin{align}
       \myvec{\frac{\vec{n}_2^{\top}}{\norm{\vec{n}_2}} - \frac{\vec{n}_3^{\top}}{\norm{\vec{n}_3}}}\vec{x}
       =
       \frac{c_2}{\norm{\vec{n}_2}}-\frac{c_3}{\norm{\vec{n}_3}}
\end{align}
on substitution we obtain 
\begin{align}
\myvec{
\frac{11}{\sqrt{122}}+\frac{1}{\sqrt{2}} & \frac{1}{\sqrt{122}}-\frac{1}{\sqrt{2}}\\
}
\vec{x}
=\frac{2}{\sqrt{2}}-\frac{38}{\sqrt{122}}
\end{align}
\end{enumerate}
\begin{figure}[H]
\includegraphics[width=\columnwidth]{solutions/1/5/1(1)/figs/anglebisector.png}
\caption{Angle bisectors plotted using python}
\label{fig:i_angbisector_py}
\end{figure}

	\item Find the intersection $\vec{I}$ of the angle bisectors of $B$ and $C$.
	\item Find the distance from $\vec{I}$ to $BC$.  
  \\
		\iffalse
\documentclass[journal,12pt,twocolumn]{IEEEtran}
\usepackage{cite}
\usepackage{amsmath,amssymb,amsfonts,amsthm}
\usepackage{algorithmic}
\usepackage{graphicx}
\usepackage{textcomp}
\usepackage{xcolor}
\usepackage{txfonts}
\usepackage{listings}
\usepackage{enumitem}
\usepackage{mathtools}
\usepackage{gensymb}
\usepackage[breaklinks=true]{hyperref}
\usepackage{tkz-euclide} % loads  TikZ and tkz-base
\usepackage{listings}
\usepackage{float}


\begin{document}
\providecommand{\pr}[1]{\ensuremath{\Pr\left(#1\right)}}
\providecommand{\prt}[2]{\ensuremath{p_{#1}^{\left(#2\right)} }}        % own macro for this question
\providecommand{\qfunc}[1]{\ensuremath{Q\left(#1\right)}}
\providecommand{\sbrak}[1]{\ensuremath{{}\left[#1\right]}}
\providecommand{\lsbrak}[1]{\ensuremath{{}\left[#1\right.}}
\providecommand{\rsbrak}[1]{\ensuremath{{}\left.#1\right]}}
\providecommand{\brak}[1]{\ensuremath{\left(#1\right)}}
\providecommand{\lbrak}[1]{\ensuremath{\left(#1\right.}}
\providecommand{\rbrak}[1]{\ensuremath{\left.#1\right)}}
\providecommand{\cbrak}[1]{\ensuremath{\left\{#1\right\}}}
\providecommand{\lcbrak}[1]{\ensuremath{\left\{#1\right.}}
\providecommand{\rcbrak}[1]{\ensuremath{\left.#1\right\}}}
\newcommand{\sgn}{\mathop{\mathrm{sgn}}}
\providecommand{\abs}[1]{\left\vert#1\right\vert}
\providecommand{\res}[1]{\Res\displaylimits_{#1}} 
\providecommand{\norm}[1]{\left\lVert#1\right\rVert}
%\providecommand{\norm}[1]{\lVert#1\rVert}
\providecommand{\mtx}[1]{\mathbf{#1}}
\providecommand{\mean}[1]{E\left[ #1 \right]}
\providecommand{\cond}[2]{#1\middle|#2}
\providecommand{\fourier}{\overset{\mathcal{F}}{ \rightleftharpoons}}
\newenvironment{amatrix}[1]{%
  \left(\begin{array}{@{}*{#1}{c}|c@{}}
}{%
  \end{array}\right)
}
%\providecommand{\hilbert}{\overset{\mathcal{H}}{ \rightleftharpoons}}
%\providecommand{\system}{\overset{\mathcal{H}}{ \longleftrightarrow}}
	%\newcommand{\solution}[2]{\textbf{Solution:}{#1}}
\newcommand{\solution}{\noindent \textbf{Solution: }}
\newcommand{\cosec}{\,\text{cosec}\,}
\providecommand{\dec}[2]{\ensuremath{\overset{#1}{\underset{#2}{\gtrless}}}}
\newcommand{\myvec}[1]{\ensuremath{\begin{pmatrix}#1\end{pmatrix}}}
\newcommand{\mydet}[1]{\ensuremath{\begin{vmatrix}#1\end{vmatrix}}}
\newcommand{\myaugvec}[2]{\ensuremath{\begin{amatrix}{#1}#2\end{amatrix}}}
\providecommand{\rank}{\text{rank}}
\providecommand{\pr}[1]{\ensuremath{\Pr\left(#1\right)}}
\providecommand{\qfunc}[1]{\ensuremath{Q\left(#1\right)}}
	\newcommand*{\permcomb}[4][0mu]{{{}^{#3}\mkern#1#2_{#4}}}
\newcommand*{\perm}[1][-3mu]{\permcomb[#1]{P}}
\newcommand*{\comb}[1][-1mu]{\permcomb[#1]{C}}
\providecommand{\qfunc}[1]{\ensuremath{Q\left(#1\right)}}
\providecommand{\gauss}[2]{\mathcal{N}\ensuremath{\left(#1,#2\right)}}
\providecommand{\diff}[2]{\ensuremath{\frac{d{#1}}{d{#2}}}}
\providecommand{\myceil}[1]{\left \lceil #1 \right \rceil }
\newcommand\figref{Fig.~\ref}
\newcommand\tabref{Table~\ref}
\newcommand{\sinc}{\,\text{sinc}\,}
\newcommand{\rect}{\,\text{rect}\,}
%%
%	%\newcommand{\solution}[2]{\textbf{Solution:}{#1}}
%\newcommand{\solution}{\noindent \textbf{Solution: }}
%\newcommand{\cosec}{\,\text{cosec}\,}
%\numberwithin{equation}{section}
%\numberwithin{equation}{subsection}
%\numberwithin{problem}{section}
%\numberwithin{definition}{section}
%\makeatletter
%\@addtoreset{figure}{problem}
%\makeatother

%\let\StandardTheFigure\thefigure
\let\vec\mathbf

\bibliographystyle{IEEEtran}


\vspace{3cm}


\textbf{1.5.4}
Find distance from $\vec{I}$ to $BC$. \\
\fi
\solution
Given: 
\begin{align}
\vec{A}=\myvec{1\\-1} \\
\vec{B}=\myvec{-4\\6} \\
\vec{C}=\myvec{-3\\-5}
\end{align}
We know incentre \begin{align}
\vec{I} = \frac{1}{\sqrt{37} + 4 + \sqrt{61}} \myvec{\sqrt{61} - 16 - 3\sqrt{37} \\ -\sqrt{61} + 24 - 5\sqrt{37}}
\end{align}
Equation of $BC$:  
\begin{align}
\vec{n} ^\top \vec{x} &= c \\ 
\myvec{11 \\ 1}^\top \vec x &= -38 
\end{align}
Distance from $\vec{I}$ to $BC$
\begin{align}
&=\frac{\abs{\vec{n}^\top \vec{I} - c}}{\norm{\vec{n}}} \\
&=\frac{\abs{\myvec{11\\1}^\top \frac{1}{\sqrt{37} + 4 + \sqrt{61}} \myvec{\sqrt{61} - 16 - 3\sqrt{37} \\ -\sqrt{61} + 24 - 5\sqrt{37}} +38}}{\norm{\myvec{11\\1}}} \\
&=\frac{\abs{\frac{\myvec{11&1}\myvec{\sqrt{61} - 16 - 3\sqrt{37} \\ -\sqrt{61} + 24 - 5\sqrt{37}}}{\sqrt{37} + 4 + \sqrt{61}}+38}}{\sqrt{122}}\\
&=\frac{\abs{\frac{10\sqrt{61}-152-38\sqrt{37}}{\sqrt{37} + 4 + \sqrt{61}}+38}}{\sqrt{122}}\\
&=\frac{48\sqrt{61}}{(\sqrt{37}+4+\sqrt{61})\sqrt{122}}\\
&=1.8968 %verified with python
\end{align}

 
	\item Repeat the above exercise for the sides $AB$ and $AC$.
	This distance is known as the {\em inradius} $r$.
	\item Draw a circle with center $\vec{I}$ and radius $r$.  $\vec{I}$ is known as the {\em incentre}.
	\item The equation of the {\em incircle} is given by 
		\begin{align}
			\label{eq:incircle}
			\norm{\vec{x}-\vec{I}}^2 = r^2
		\end{align}
		Using the parameteric equation of $BC$, verify that $BC$ intersects the incircle at exactly one point. $BC$ is defined to be a {\em tangent} to the incircle.
		\\
		\solution
Let 
\begin{align}
\vec{x} &= \vec{B} + k{\vec{m}}\label{eq:8}
\end{align}
Substituting \eqref{eq:8} in \eqref{eq:incircle}
\begin{align}
  \norm{ \vec{B} + k{\vec{m}}- \vec{I} }^2 &= r^2 \\
\implies   \brak{\vec{B} + k{\vec{m}}- \vec{I}}^{\top} \brak{\vec{B} + k{\vec{m}}- \vec{I}} &= r^2 
\\
	\text{or, }
	k^2\norm{\vec{m}}^2 +2k{\vec{m}^{\top}}\brak{{\vec{B}-\vec{I}}}+\norm{\vec{B}-\vec{I}}^2 &= 0
	\label{eq:incircle-disc}
\end{align}
It can be easily verified that 
\begin{align}
\cbrak{2\vec{m}^{\top}\brak{\vec{B}-\vec{I}}}^2
= 
	\norm{\vec{m}}^2\norm{\vec{B}-\vec{I}}^2
\end{align}
which implies that the discriminant of 
	\eqref{eq:incircle-disc}
	is 0.  Thus, BC intersects the circle at only one point.


	\item Find the {\em point of contact} $\vec{D}_3$ where $BC$ touches the incircle.
		\\
		\solution
Since
	\eqref{eq:incircle-disc}
	has only one root, 
\begin{align}
k=-\frac{\vec{m}^{\top}\brak{\vec{B}-\vec{I}}}{{\ \norm{\vec{m}}^2}}
\end{align}
Substituing the above in 
\eqref{eq:8},
\begin{align}
	\vec{D_{3}} = \vec{B} -\frac{\vec{m}^{\top}\brak{\vec{B}-\vec{I}}}{{\ \norm{\vec{m}}^2}} \vec{m}
\end{align}



  \item Find the other points of contact $\vec{E}_3$ and $\vec{F}_3$.
  \\
		\iffalse
\let\negmedspace\undefined
\let\negthickspace\undefined
\documentclass[journal,12pt,twocolumn]{IEEEtran}
\usepackage{cite}
\usepackage{amsmath,amssymb,amsfonts,amsthm}
\usepackage{algorithmic}
\usepackage{graphicx}
\usepackage{textcomp}
\usepackage{xcolor}
\usepackage{txfonts}
\usepackage{listings}
\usepackage{enumitem}
\usepackage{mathtools}
\usepackage{gensymb}
\usepackage[breaklinks=true]{hyperref}
\usepackage{tkz-euclide} 
\usepackage{listings}
\usepackage{gvv}
%
%\usepackage{setspace}
%\usepackage{gensymb}
%\doublespacing
%\singlespacing

%\usepackage{graphicx}
%\usepackage{amssymb}
%\usepackage{relsize}
%\usepackage[cmex10]{amsmath}
%\usepackage{amsthm}
%\interdisplaylinepenalty=2500
%\savesymbol{iint}
%\usepackage{txfonts}
%\restoresymbol{TXF}{iint}
%\usepackage{wasysym}
%\usepackage{amsthm}
%\usepackage{iithtlc}
%\usepackage{mathrsfs}
%\usepackage{txfonts}
%\usepackage{stfloats}
%\usepackage{bm}
%\usepackage{cite}
%\usepackage{cases}
%\usepackage{subfig}
%\usepackage{xtab}
%\usepackage{longtable}
%\usepackage{multirow}
%\usepackage{algorithm}
%\usepackage{algpseudocode}
%\usepackage{enumitem}
%\usepackage{mathtools}
%\usepackage{tikz}
%\usepackage{circuitikz}
%\usepackage{verbatim}
%\usepackage{tfrupee}
%\usepackage{stmaryrd}
%\usetkzobj{all}
%    \usepackage{color}                                            %%
%    \usepackage{array}                                            %%
%    \usepackage{longtable}                                        %%
%    \usepackage{calc}                                             %%
%    \usepackage{multirow}                                         %%
%    \usepackage{hhline}                                           %%
%    \usepackage{ifthen}                                           %%
  %optionally (for landscape tables embedded in another document): %%
%    \usepackage{lscape}     
%\usepackage{multicol}
%\usepackage{chngcntr}
%\usepackage{enumerate}

%\usepackage{wasysym}
%\documentclass[conference]{IEEEtran}
%\IEEEoverridecommandlockouts
% The preceding line is only needed to identify funding in the first footnote. If that is unneeded, please comment it out.

\newtheorem{theorem}{Theorem}[section]
\newtheorem{problem}{Problem}
\newtheorem{proposition}{Proposition}[section]
\newtheorem{lemma}{Lemma}[section]
\newtheorem{corollary}[theorem]{Corollary}
\newtheorem{example}{Example}[section]
\newtheorem{definition}[problem]{Definition}

\newcommand{\BEQA}{\begin{eqnarray}}
\newcommand{\EEQA}{\end{eqnarray}}
\newcommand{\define}{\stackrel{\triangle}{=}}
\theoremstyle{remark}
\newtheorem{rem}{Remark}


\begin{document}


\bibliographystyle{IEEEtran}


\vspace{3cm}

\title{
	Question 1.5.9
}

\author{
	EE22BTECH11054 - Umair Parwez
}	

\maketitle
\newpage


\renewcommand{\thefigure}{\theenumi}
\renewcommand{\thetable}{\theenumi}


\textbf{Question 1.5.9:}

Given triangle $ABC$ with vertices, 
\begin{align}
	\vec{A} = \myvec{1\\ -1}, \vec{B} = \myvec{-4\\ 6}, \vec{C} = \myvec{-3\\ -5}
\end{align}
Find the points of contact, $\vec{E}_3$ and $\vec{F}_3$, of the incircle with sides $AC$ and $AB$ respectively.
\fi
\textbf{Solution:}

Required to find points of contact, $\vec{E}_3$ and $\vec{F}_3$, of incircle with sides $AC$ and $AB$ respectively.
From previous questions we know the coordinates of the incircle are : 
\begin{align}
	\vec{I} &= 
	\myvec {
		\frac{-53-11\sqrt{37}+7\sqrt{61}+\sqrt{2257}}{12} \\ 
		\frac{5-\sqrt{37}+5\sqrt{61}-\sqrt{2257}}{12}
	} \\
	&= \myvec{-1.47756 \\ -0.79495}
\end{align}
Radius of incircle is :
    \begin{align}
		r &= \frac{185+41\sqrt{37}-37\sqrt{61}-\sqrt{2257}}{6\sqrt{74}} \\
		&= 1.89689
	\end{align}
Equation of incircle is : 
\begin{align}
	{\norm{ \vec{x}-\vec{I} }}^2 = {r}^2 \label{eq:circle}
\end{align}
points $\vec{A}$, $\vec{B}$ and $\vec{C}$ are : 
\begin{align}
	\vec{A} = \myvec{
		1\\
		-1
	}, 
	\vec{B} = \myvec{
		-4\\
		6
	}, 
	\vec{C} = \myvec{
		-3\\
		-5
	}
\end{align}
Parametric equation of any line is of the form :
\begin{align}
	\vec{x} = \vec{A} + k\vec{m} \label{eq:param}
\end{align}
We can substitute \eqref{eq:param} in \eqref{eq:circle}, and we get : 
\begin{align}
	{\norm{ \vec{A}+k\vec{m}-\vec{I} }}^2 &= {r}^2 \\
	\implies (k\vec{m} + (\vec{A} - \vec{I}))(k\vec{m} + (\vec{A} - \vec{I})) &= {r}^2 \\
	\implies k^2 \norm{\vec{m}}^2 + 2k\vec{m}^{\top} (\vec{A}-\vec{I}) + \norm{\vec{A}-\vec{I}}^2 &= {r}^2 
\end{align}
The discriminant of the above quadratic is,
\begin{align}
	\Delta = 4(\vec{m}^{\top}(\vec{A}-\vec{I}))^2 - 4\norm{\vec{m}}^2(\norm{\vec{A}-\vec{I}}^2 - r^2) \label{eq:Delta}
\end{align} 
If,
\begin{align}
	\Delta = 0
\end{align}
Then the line given by \eqref{eq:param} is tangent to the incircle and the value of $k$ will be given by :
\begin{align}
	k = -\frac{\vec{m}^{\top}(\vec{A} - \vec{I})}{\norm{\vec{m}}^2} \label{eq:k}
\end{align}
Upon substituting $k$ back into \eqref{eq:param}, we will obtain the point of contact of the line with the incircle.
\begin{enumerate}
	\item Finding $\vec{E}_3$ :\\
		Parametric equation of $AC$ is,
		\begin{align}
			\vec{x} = \vec{A} + k\vec{m} \label{eq:AC}
		\end{align}
		where,
		\begin{align}
			\vec{m} = \vec{A} - \vec{C}
		\end{align}	
		First we confirm value of $\Delta$ for $AC$,
		\begin{align}
			\Delta &= 4(\vec{m}^{\top}(\vec{A}-\vec{I}))^2 - 4\norm{\vec{m}}^2(\norm{\vec{A}-\vec{I}}^2 - r^2)\\	
			&= 4(9.09004)^2 - 4(32)(6.18035-3.59819) \\
			&= 0
		\end{align}
		Therefore, $AC$ is tangent to the incircle, and value of $k$ is given by, 
		\begin{align}
			k &= -\frac{\vec{m}^{\top}(\vec{A} - \vec{I})}{\norm{\vec{m}}^2} \\
			&= -\frac{(\vec{A}-\vec{C})^{\top}(\vec{A} - \vec{I})}{\norm{\vec{A}-\vec{C}}^2}
		\end{align}
		Substituting the values, we get,
		\begin{align}
			k = \frac{-4-\sqrt{37}+\sqrt{61}}{2} \label{k:AC} 
		\end{align}
		Substituting \eqref{k:AC} into \eqref{eq:AC}, we get the value of $\vec{E}_3$,
		\begin{align}
			\vec{E}_3 = \myvec{
				\frac{-2-\sqrt{37}+\sqrt{61}}{2}\\ 
				\frac{-6-\sqrt{37}+\sqrt{61}}{2}
			}
		\end{align}
	\item Finding $\vec{F}_3$ :\\
		Parametric equation of $AB$ is,
		\begin{align}
			\vec{x} = \vec{A} + k\vec{m} \label{eq:AB}
		\end{align}
		where,
		\begin{align}
			\vec{m} = \vec{A} - \vec{B}
		\end{align}	
		First we confirm value of $\Delta$ for $AB$,
		\begin{align}
			\Delta &= 4(\vec{m}^{\top}(\vec{A}-\vec{I}))^2 - 4\norm{\vec{m}}^2(\norm{\vec{A}-\vec{I}}^2 - r^2)\\	
			&= 4(13.82315)^2 - 4(74)(6.18035-3.59819) \\
			&= 0
		\end{align}
		Therefore, $AB$ is tangent to the incircle, and value of $k$ is given by, 
		\begin{align}
			k &= -\frac{\vec{m}^{\top}(\vec{A} - \vec{I})}{\norm{\vec{m}}^2} \\
			&= -\frac{(\vec{A}-\vec{B})^{\top}(\vec{A} - \vec{I})}{\norm{\vec{A}-\vec{B}}^2}
		\end{align}
		Substituting the values, we get,
		\begin{align}
			k = \frac{-37-4\sqrt{37}+\sqrt{2257}}{74} \label{k:AB} 
		\end{align}
		Substituting \eqref{k:AB} into \eqref{eq:AB}, we get the value of $\vec{F}_3$,
		\begin{align}
			\vec{F}_3 = \myvec{
				\frac{-111-20\sqrt{37}+5\sqrt{2257}}{74}\\ 
				\frac{185+28\sqrt{37}-7\sqrt{2257}}{74}
			}
		\end{align}
\end{enumerate}
Diagram is shown below,
\begin{figure}[h]
	\centering
	\includegraphics[width=\columnwidth]{./figs/Diagram.png}
	\caption{Points of contact of incircle}
	\label{fig:Incircle}
\end{figure}




		\fi
\end{enumerate}

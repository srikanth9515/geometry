%\renewcommand{\theequation}{\theenumi}
%\begin{enumerate}[label=\arabic*.,ref=\theenumi]
The matrix of the veritices of the triangle is defined as
		\begin{align}
			\vec{P} = \myvec{\vec{A} & \vec{B} &\vec{C}}
		\end{align}
\begin{enumerate}[label=\thesection.\arabic*.,ref=\thesection.\theenumi]
\numberwithin{equation}{enumi}
\item Obtain the direction matrix of the sides of $\triangle ABC$
	defined as 
		\begin{align}
		\vec{M} = 	\myvec{\vec{A}-\vec{B} & \vec{B}-\vec{C} & \vec{C}-\vec{A}}
		\end{align}
	\\
		\solution 

		\begin{align}
			\vec{M} = \myvec{\vec{A}-\vec{B} & \vec{B}-\vec{C} & \vec{C}-\vec{A}}
			&= 
			\myvec{\vec{A} & \vec{B} &\vec{C}}
			\myvec{1 & 0 & -1 \\ -1 & 1 & 0 \\ 0 & -1 & 1}
			\\
			&=\vec{P}\vec{Q}
		\end{align}
		where 
		\begin{align}
\vec{Q}= 
			\myvec{1 & 0 & -1 \\ -1 & 1 & 0 \\ 0 & -1 & 1}
		\end{align}
		is known as a {\em circulant} matrix.  Note that the 2nd and 3rd row of the above matrix are circular shifts of the 1st row.
	\item Obtain the normal matrix  of the sides of $\triangle ABC$
		\\
		\solution Considering the roation matrix
		\begin{align}
			\vec{R}  = \myvec{0 & -1 \\ 1 & 0},
		\end{align}
		the normal matrix is obtained as
		\begin{align}
			\vec{N} = \vec{R}\vec{M}\vec{Q}
		\end{align}

	\item Obtain $a, b, c$.
		\\
		\solution The sides vector is obtained as
		\begin{align}
			\vec{d} = \sqrt{\text{diag}(\vec{M}^{\top}\vec{M})}
		\end{align}
	\item Obtain the line equations
\end{enumerate}

\iffalse
\begin{figure}[!ht]
\centering
\resizebox{\columnwidth}{!}{\input{./figs/triangle/tri_right_angle.tex}}
\caption{}
\label{fig:tri_right_angle_area}	
\end{figure}
\fi
\renewcommand{\theequation}{\theenumi}
\begin{enumerate}[label=\thesection.\arabic*.,ref=\thesection.\theenumi]
\numberwithin{equation}{enumi}


\item
	Show that 
	\begin{equation}
	\frac{\sin A}{a} = \frac{\sin B}{b} = \frac{\sin C}{c}
	\end{equation}

\solution Fig. \ref{fig:tri_sss} can be suitably modified to obtain 
\begin{align}
ar\brak{\Delta ABC} = 
\frac{1}{2}ab\sin C = \frac{1}{2}bc\sin A = \frac{1}{2}ca\sin B
\end{align}
Dividing the above by $abc$, we obtain
	\begin{equation}
\label{eq:tri_sin_form}
	\frac{\sin A}{a} = \frac{\sin B}{b} = \frac{\sin C}{c}
	\end{equation}
This is known as the sine formula.	
%
%
\item Show that 
%
\begin{align}
\label{eq:trig_id_sin_inc}
\alpha > \beta \implies \sin \alpha > \sin \beta
\end{align}
%

\begin{figure}[!ht]
	\begin{center}
		
		%\includegraphics[width=\columnwidth]{./figs/fig:tri_sin_inc}
		%\vspace*{-10cm}
		\resizebox{\columnwidth}{!}{\input{./figs/triangle/tri_sin_inc.tex}}
	\end{center}
	\caption{}
	%\caption{$\sin \brak{\theta_1+\theta_2} = \sin\theta_1\cos\theta_2 + \cos\theta_1\sin\theta_2$}
	\label{fig:tri_sin_inc}	
\end{figure}
\solution In Fig. \ref{fig:tri_sin_inc}, 	
%
\begin{align}
ar\brak{\triangle ABD} &< ar \brak{\triangle ABC}
\\
\implies \frac{1}{2}lc \sin \theta_1 &<  \frac{1}{2}ac \sin \brak{\theta_1 + \theta_2 }
\\
\implies \frac{l}{a} &< \frac{\sin \brak{\theta_1 + \theta_2 }}{\sin \theta_1}
\\
\text{or, } 1 < \frac{l}{a} &< \frac{\sin \brak{\theta_1 + \theta_2 }}{\sin \theta_1}
\\
\implies \frac{\sin \brak{\theta_1 + \theta_2 }}{\sin \theta_1} > 1
\end{align}
%
from Theorem \ref{them:hyp_largest}. This proves \eqref{eq:trig_id_sin_inc}.
%From \eqref{eq:trig_id_sum_diff3},
%%
%\begin{multline}
% \sin \theta_1 - \sin \theta_2 = 2\sin\brak{\frac{\theta_1-\theta_2}{2}}
%\\
%\times \cos\brak{\frac{\theta_1+\theta_2}{2}} > 0, \because \theta_1-\theta_2 > 0
%\end{multline}
%
\item In a triangle, the side opposite the greater angle is greater.
\begin{figure}[!ht]
	\begin{center}
			\resizebox{\columnwidth}{!}{\input{./figs/quad/tri_ang_side.tex}}
	\end{center}
	\caption{Side opposite the greater angle is greater}
	\label{fig:tri_ang_side}	
\end{figure}
\\
\solution In Fig. 	\ref{fig:tri_ang_side},	let
%
\begin{align}
\angle B > \angle C
\end{align}
%
Then, using the sine formula,
%
\begin{align}
\frac{\sin B}{b} &=\frac{\sin C}{c}
\\
\implies   \frac{\sin B}{\sin C} &= \frac{b}{c} > 1
\end{align}
using \eqref{eq:trig_id_sin_inc}.


\end{enumerate}
